\documentclass{article}
\usepackage{graphicx}
\usepackage{amsmath}

%%%%%%%%%%%%%%%%%%%%%%%%%%%%%%%%%%%%%%%%%%%%%%%%%%%%%%%%%%%%%%%%%%%%%%
\begin{document}

\title{Notes for research HALEU enrichment and demand}
\author{Amanda Bachmann}
\maketitle

\section{Uranium Enrichment}
\cite{harding_12_2016} provides an overview of the enrichment process, 
including some demand scenarios out to 2030. Does not specify if 
LEU or HALEU. Section 12.2.3 looks at the current suppliers of enriched 
uranium, including URENCO, AREVA (now Framatome). Enrichment Technology 
Company, Russia, China, Brazil, Japan. and others with capabilities. 
Section 12.2.6 discusses the supply and demand system for enriched U. 
12.7 discusses the challenges of enriching reprocessed U. U-232 will 
release hard gammas in its decay chain, U-234 and U-236 are neutron 
absorbers. All will be increased in relative concentration with U-235. 
Dose from the fuel is a much larger concern when using reprocessed 
material. 

According to NEI white paper (2018) there are 3 commercial enrichment 
processes developed for use: gaseous diffusion, gas centrifuge, and laser 
separation. "There is no operational enrichment facility in the 
United States that can currently produce uranium enrichments of greater
than 5\%." Only source for U above 5\% is downblending government 
owned HEU--surplus weapons grade material, reprocessed naval reactor
fuel. Down-blending is done in Erwin and Oak Ridge. Naval reactor fuel 
could be a "stop-gap strategy... but cannot be relied upon as a long-
term fuel source." There is one enrichment facility currently 
operating, Eunuce, NM, with two licenses granted by the NRC [check 
NRC website, licenses given to Centrus]. GE-Hitachi has pursued 
licensing for a laser enrichment facility. Licensing for a new 
enrichment plant is expected to take 2.5 years, with an extra 0.5 
years for a mandatory hearing to take place. Modifying a plant to 
increase to HALEU levels is expected to take 12-18 months with an 
environmental review. 

\section{HALEU Demand}
White paper from NEI (2018) suggests that if the US can't develop 
enough of a supply of HALEU, then the first few reactors will have 
to get their fuel from a foreign sources. If a full supply base 
is not established, then the US won't be able to provide fuel 
with any reactor designs we supply to other countries. Statuatory
requirements for SNM are found in 10 CFR Part 70 -- will need license 
under this to construct facility to enrich above 5\%. Will also need 
to have HALEU for research and demonstration reactors. 

\section{HALEU Enrichment}

\section{HALEU Fabrication}
NEI white paper (2018) highlights that there may need to be changes 
in the fuel fabrication step of the fuel cycle since many of the 
advanced reactors don't use UO$_2$ fuel. May require deconversion 
of UF$_6$ to U metal, or another form (like TRISO). Other countries 
have HALEU fuel production facilities that may provide some guidance. 
An NRC license amendment would be required to allow any of the 
current fuel fab facilities to perform HALEU fuel fab. Suggests 
that current supply of U ore and yellowcake would be sufficient for 
HALEU reactor generation. There is a surplus of UF$_6$, so Honeywell 
has been idle for a few years. 

\section{HALEU as UNF}
INL has a technical report about the isotopic content of HALEU to 
"assess the feasibility of reusing HALEU recovered from the treatement 
of irradiated EBR-II fuel" \cite{vaden_isotopic_2018}. The fuel undegoes 
an electrometallurgical treatement process (EMT). The measurements
are done on fuel that is U metal downblended to less that 20\% U-235.
Generated SNF compositions with ORIGEN, used isotopic distributions from 
MTG, their mass tracking system. Used together to estimate the masses of 
non-measured isotopes. Compositions listed are considered bouding cases. 

EBR-II fuel was HE, sodium-bonded, metallic U as driver-fuel 
\cite{patterson_haleu_2019}. Once the sodium has been stripped, leaving 
a solid U ingot, the fuel can be downblended for use as HALEU fuel.
INL is doing reseach into how EMT can be improved to increase the purity 
of the material used for fuel fabrication. I think they are mostly 
referring to the removal of fission products, not the removal of bad 
U/Pu isotopes or other actinides that are chemically similar to U. 
Potential of 3.86 MT of HALEU feedstock currently, may have 6 MT more. 
Report discusses ways to recast the U ingots into smaller sizes for 
potential use outside of a glovebox. Recasted ingots will have a U 
balance of 1.2 kg. Some of the reguli show elevated levels of Cs and 
Pu, indicating successful separation of contaminants. Handle the U 
ingots in closed metal containers to prevent oxidation. Compare the dose 
rate from slags and reguli to qualitatively estimate the purity.
"dose rate of potential HALEU from EBR-II in ingot form can be reduced 
significantly by drip casting into a regulus form." Will look into 
upblending the U-235 content for casting. 

\section{Reactors}

\section{MC\&A for HALEU}
NEI white paper (2018) states that license applications for facilities
to handle and process HALEU must outline their saefguards and MC\&A 
strategies. Regulatory requirements are in 10 CFR 74, draft regulatory 
guidance is in NUREG-2159

\bibliographystyle{ieeetr}
\bibliography{notes-ref}
\end{document}