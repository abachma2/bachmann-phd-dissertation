\documentclass{report}
\usepackage[margin=1in]{geometry}
\usepackage{multirow}
\usepackage{lineno}
\usepackage{xspace}
\usepackage{threeparttable}
\usepackage{caption}
\usepackage{subcaption}
\usepackage{outline}
\usepackage{placeins}
\usepackage{booktabs} % nice rules (thick lines) for tables
\usepackage{microtype} % improves typography for PDF
\usepackage{hhline}
\usepackage{amsmath}
\usepackage{tabularx}

\newcommand{\Cyclus}{\textsc{Cyclus}\xspace}%
\newcommand{\Cycamore}{\textsc{Cycamore}\xspace}%
\usepackage[acronym,shortcuts]{glossaries}
\include{acros}


\begin{document}

\begin{outline}
\item Working title: Fuel cycle impacts of deploying \gls{HALEU}-fueled reactors

\item Introduction
\begin{outline}
    \item Purpose of the work: investigate the effect of deploying HALEU-fueled 
          advanced reactors on the nuclear fuel cycle in the US 
    \item Scope:
    \begin{outline}
        \item US facilities
        \item select advanced reactors: USNC MMR, X-energy Xe-100, NuScale VOYGR
        \item Front-end and back-end of the fuel cycle
    \end{outline}
    \item Benefits of using HALEU for reactors -- why we care about these reactors
\end{outline}

\item Lit Review
\begin{outline}
    \item The nuclear fuel cycle
    \begin{outline}
          \item Once-through vs recycle
          \item Enrichment facility/SWU calculations
          \item Recycling processe
    \end{outline}
    
    \item Fuel Cycle simulators
    \begin{outline}
        \item \Cyclus \cite{huff_fundamental_2016}
        \begin{outline}
              \item \Cycamore \cite{scopatz_cyclus_2015}
        \end{outline}
        \item DYMOND \cite{feng_standardized_2016}
        \item Use and verification 
        \begin{outline}
            \item Verification \cite{feng_standardized_2016,bae_standardized_2019}
            \item SImulators comparison \cite{djokic_application_2015}
        \end{outline}
    \end{outline}
    \item Fuel Cycle modeling
    \begin{outline}
          \item \gls{DOE} Evaluation \& screening \cite{wigeland_nuclear_2014}
          \begin{outline}
                \item 
          \end{outline}
    \end{outline}
\end{outline}

\item Modeling material flows -- once through fuel cycles
\begin{outline}
    \item Modeling individual reactor transitions
    \begin{outline}
        \item MMR 
        \begin{outline}
            \item No growth transitions
            \item 1\% growth transitions
        \end{outline}
        \item Xe-100
        \begin{outline}
            \item No growth transitions
            \item 1\% growth transitions
        \end{outline}
    \end{outline}
    \item Modeling multi-reactor transitions 
    \begin{outline}
        \item Xe-100 and MMR
        \begin{outline}
            \item No growth transitions
            \item 1\% growth transitions
        \end{outline} 
        \item Xe-100 and VOYGR 
        \begin{outline}
            \item No growth transitions
            \item 1\% growth transitions
        \end{outline}
        \item MMR and VOYGR 
        \begin{outline}
            \item No growth transitions
            \item 1\% growth transitions
        \end{outline}
        \item Xe-100, MMR, and VOGYR 
        \begin{outline}
            \item No growth transitions
            \item 1\% growth transitions
        \end{outline}
    \end{outline}
    \item Sensitivity analysis and optimization
\end{outline}

\item Model fuel cycle with recycle 
\begin{outline}
    \item Xe-100 and MMR with fuel going to LWRs
\end{outline}

\item Downblending effects on neutronics

\item Conclusions
\end{outline}


\bibliography{bibliography}
\bibliographystyle{ieeetr}
\end{document}