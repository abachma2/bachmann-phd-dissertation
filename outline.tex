\documentclass{report}
\usepackage[margin=1in]{geometry}
\usepackage{multirow}
\usepackage{lineno}
\usepackage{xspace}
\usepackage{threeparttable}
\usepackage{caption}
\usepackage{subcaption}
\usepackage{outline}
\usepackage{placeins}
\usepackage{booktabs} % nice rules (thick lines) for tables
\usepackage{microtype} % improves typography for PDF
\usepackage{hhline}
\usepackage{amsmath}
\usepackage{tabularx}

\newcommand{\Cyclus}{\textsc{Cyclus}\xspace}%
\newcommand{\Cycamore}{\textsc{Cycamore}\xspace}%
\usepackage[acronym,shortcuts]{glossaries}
\include{acros}


\begin{document}
\noindent Working title: Fuel cycle impacts of deploying \gls{HALEU}-fueled reactors
\begin{outline}
\item Introduction
\begin{outline}
    \item Purpose of the work: investigate the effect of deploying HALEU-fueled 
          advanced reactors on the nuclear fuel cycle in the US 
    \item Scope:
    \begin{outline}
        \item US facilities
        \item select advanced reactors: USNC MMR, X-energy Xe-100, NuScale VOYGR
        \item Front-end and back-end of the fuel cycle
    \end{outline}
    \item Motivations
    \begin{outline}
        \item Benefits of using HALEU for reactors 
        \item Changing the fuel form affects fuel cycle dynamics
    \end{outline}
    \item Goals
    \begin{outline}
        \item understand how deploying HALEU reactors affects resource demand
        \item understand which components of the fuel cycle are most sensitive to HALEU deployments
        \item understand how implementing recycling with HALEU reactors affects the fuel cycle
        \item understand how possible avenues to obtain fuel for HAELU reactors can affect reactor performance
    \end{outline}
\end{outline}

\item Lit Review
\begin{outline}
    \item The nuclear fuel cycle
    \begin{outline}
          \item Once-through vs recycle \cite{tsoulfanidis_nuclear_2013}
          \item Enrichment facility/SWU calculations \cite{tsoulfanidis_nuclear_2013}
          \begin{outline}
              \item classifications of uranium, LEU vs HEU vs HALEU
          \end{outline}
          \item Recycling processes \cite{tsoulfanidis_nuclear_2013}
          \begin{outline}
              \item overview of aqueous reprocessing 
              \item Known changes to LWR fuel cycle by recycling 
              \item 
          \end{outline}
    \end{outline}
    
    \item Fuel Cycle simulators
    \begin{outline}
        \item Why we use them, their benefits
        \item why multiples have been created
        \item ideal functionalities and capcbilities \cite{huff_next_2010,brown_identification_2016}
        \item uses of fuel cycle simulators
        \begin{outline}
            \item \gls{DOE} Evaluation \& screening \cite{wigeland_nuclear_2014}
                \begin{outline}
                      \item Differences in EG 01 and EG 02 
                \end{outline}
            \item Effects of changing from 5\% to 7\% for PWR \cite{burns_reactor_2020}
            \item EG29 analysis \cite{sunny_transition_2015}
            \item verification \cite{feng_standardized_2016}
        \end{outline}
        \item sensitivity studies
        \item \Cyclus \cite{huff_fundamental_2016}
        \begin{outline}
              \item basic fundamentals 
              \item \Cycamore \cite{scopatz_cyclus_2015}
              \item addresses many of the things brought up by \cite{huff_next_2010}
              \item comparison to other codes \cite{djokic_application_2015}
              \item verification \cite{bae_standardized_2019}
        \end{outline}
    \end{outline}
    \item Reactors
    \begin{outline}
        \item USNC MMR \cite{hussain_advances_2018}
        \item X-energy Xe-100 \cite{harlan_x-energy_2018}
        \item NuScale VOYGR
    \end{outline}
    
\end{outline}

\item Material requirements -- Once through fuel cycles
\begin{outline}
    \item Methodology
    \item Scenario Definitions
    \item Results
    \begin{outline}
        \item Reactor deployment
        \begin{outline}
            \item No growth scenarios
            \item 1\% growth scenarios
        \end{outline}
        \item Uranium resources
        \begin{outline}
            \item No growth scenarios
            \item 1\% growth scenarios
        \end{outline}
        \item SWU capacity
        \begin{outline}
            \item No growth scenarios
            \item 1\% growth scenarios
        \end{outline}
    \end{outline}
\end{outline}

\item Sensitivity analysis and optimization
\begin{outline}
    \item Methodology
    \item Results
\end{outline}

\item Model fuel cycle with recycle 
\begin{outline}
    \item Methodology
    \item Scenario Definitions
    \item Results
\end{outline}

\item Downblending effects on neutronics

\item Conclusions
\end{outline}


\bibliography{bibliography}
\bibliographystyle{ieeetr}
\end{document}