\documentclass{article}

\usepackage{multirow}
\usepackage{lineno}
\usepackage{xspace}
\usepackage{threeparttable}
\usepackage{caption}
\usepackage{subcaption}
\usepackage[hidelinks]{hyperref}
\usepackage[margin=1in]{geometry}
\modulolinenumbers[5]


%% `Elsevier LaTeX' style
\bibliographystyle{elsarticle-num}
%%%%%%%%%%%%%%%%%%%%%%%

%%%% packages and definitions (optional)
\usepackage{placeins}
\usepackage{booktabs} % nice rules (thick lines) for tables
\usepackage{microtype} % improves typography for PDF
\usepackage{hhline}
\usepackage{amsmath}

\usepackage{threeparttable, tablefootnote}

\usepackage{tabularx}


%% Special typesetting for Cyclus
\newcommand{\Cyclus}{\textsc{Cyclus}\xspace}%
\newcommand{\Cycamore}{\textsc{Cycamore}\xspace}%
\usepackage[acronym,shortcuts]{glossaries}
\include{acros}

%\makeglossary

\begin{document}
    \section{Introduction}
    

    \section{Review}
    Fuel cycle modeling has become an important part of understanding 
    transition scenarios. It entails modeling the flow of materials 
    from one fuel cycle facility to another, and the deployment of 
    facilities to meet the demands of the scenario modeled, such as
    reactors to meet a power demand or enrichment facilities to meet 
    the fuel demand of the reactors. To help address the 
    variety of applications 
    of fuel cycle models, multiple fuel cycle mdeling programs have 
    been developed. 

    This work discusses and compares some of the fuel cycle simulation 
    software that has been developed. 
    
    \subsection{DYMOND}
    The \gls{DYMOND} is the primary 
    system dynamics model used by AFCI to evaluate fuel cycle scenarios of 
    advanced nuclear energy systems \cite{yacout_visionverifiable_2006}.
    It is run using the \textit{iThink} software with Microsoft Excel 
    templates as the software inputs and outputs \cite{feng_standardized_2016}.
    Inputs to DYMOND include reactor and fuel characteristics, fuel cycle
    facilities, possible material flow pathways, and the power demand of the 
    scenario \cite{feng_standardized_2016}. 
    \gls{DYMOND} utilizes recipes to define reactor fuel inputs and outputs 
    \cite{yacout_visionverifiable_2006}. Current recipes available are limited 
    to specific reactor designs and burnups \cite{yacout_visionverifiable_2006}.

    \subsection{VISION}
    \Ac{VISION} was developed as part 
    of the the U.S. Department of Energy's Advanced Fuel Cycle Initiative 
    (AFCI) to help meet the initiative's objectives 
    \cite{yacout_visionverifiable_2006}. VISION was developed to 
    calculate the quantitative metrics to meet the explicit goals of the 
    AFCI. VISION calculates metrics to describe the characteristics and 
    consequences of the modeled fuel cycle. These metrics are grouped by the 
    AFCI program objectives and include waste management, proliferation 
    resistance, energy recovery, safety, and economics 
    \cite{yacout_visionverifiable_2006}. VISION is designed to simualte the 
    entire fuel cycle, but it is not designed to manage the fuel cycle, such 
    as tracking individual fuel assemblies \cite{yacout_visionverifiable_2006}.

    VISION is comprised of multiple modules that evaluate each mass flow and 
    metric \cite{yacout_visionverifiable_2006}. VISION doesn't do direct 
    neutronics calculations, it uses parameters output from external 
    neutronics calculations \cite{yacout_visionverifiable_2006}. To account 
    for changes in isotopic concentration (especially the short-lived 
    isotopes) VISION includes isotopic decay modeling. This assists in 
    understanding the waste forms in the fuel cycle \cite{yacout_visionverifiable_2006}.

    Economics modeling is implemented to closely match the AFCI Cost Basis 
    Report \cite{yacout_visionverifiable_2006}. 

    (Pulled directly from \cite{yacout_visionverifiable_2006} The proposed VISION 
    model would be used for the following:
    \begin{itemize}
        \item Evaluating the range of options against the range of objectives
        \item Examining the implications of different mixes of reactors, impact of deployment of
        different technologies, as well as potential “exit” or “off ramp” approaches to phase out
        technologies if the need arises.
        \item Examining timing issues of reactor deployment, reprocessing against waste generation
        and repository needs.
        \item Evaluating the capability of various reactor systems to handle transmutation, including
        extended burn-up of plutonium in Light Water Reactors (LWRs) and gas-cooled reactors,
        potential for destroying minor actinides in LWRs, and consumption of transuranics in fast
        reactors and accelerator driven systems.
        \item Assessing the benefits of advanced fuel cycles to reduce the need for additional
        geological waste repositories and more efficiently use the first repository.
        \item Performing dynamic simulations of fuel cycles to quantify infrastructure requirements
        and identify key trade-offs between alternatives.
        \item Evaluating creative solutions to make the nuclear fuel cycle cost competitive.
        \item Evaluating repository performance for characteristics such as volume, mass, and heat
        load; comparing various fuel cycles, reactor facility requirements, life cycle costs, and
        repository savings.
    \end{itemize}

    Can provide time-dependent analysis, as shown in \cite{piet_dynamic_2011}, 
    which simulated the transition from LWRs to advanced reactors with different 
    forms of recycling. 



    \subsection{Cyclus}
    \Cyclus is an agent-based fuel cycle simulator \cite{huff_fundamental_2016}. 
    It employs a modular architecture to model dynamic fuel cycles 
    \cite{huff_fundamental_2016}.

    Similar to VISION, \Cyclus can also do economics modeling. 

    

    \subsection{ORION}
    ORION is a fuel cycle simulator developed by the UK National Nuclear 
    Laboratory to track material movement through a user-defined nuclear 
    fuel cycle, model decay of radionuclides, and model transmutation of fuel
    \cite{gregg_analysis_2012}. The modeling is accurate enough to provide 
    high-fidelity radioactive decay and tracking of over 2000 radionuclides
    \cite{feng_standardized_2016}. 
    
    ORION has the ability to model each facility 
    as individual objects, which can be grouped together to represent small-scale 
    facilities. Can automate facility deployment to meet electricity demand
    \cite{feng_standardized_2016}, simialar to what has been developed through 
    the \Cycamore archetypes for \Cyclus \cite{scopatz_cyclus_2015}.  

    ORION can also simulate in-core irradiation of fuel, uisng externally 
    generated lattice physics calculations prior to the ORION analysis
    \cite{feng_standardized_2016}. This is different from \Cyclus, 
    DYMOND, and VISION, which all use defined recipes for fuel before 
    and after the reactor. 

    ORION has been used to evaluate the transition from the current fleet of 
    Light Water Reactors in the U.S. to a fleet of sodium fast reactors with 
    continuius recycling \cite{sunny_transition_2015}.

    
    \subsection{Comparison}
    There was a multi-lab effort to compare the above mentioned fuel cycle 
    simulators. Results from \Cyclus are in \cite{djokic_application_2015}


    \subsubsection{Verification}

    Verification of fuel cycle models has been performed
    \cite{feng_standardized_2016,bae_standardized_2019}. Verification efforts
    compared the performance of DYMOND, VISION, ORION, MARKAL 
    \cite{feng_standardized_2016} and \Cyclus later \cite{bae_standardized_2019}, 
    making adjustments to the codes as needed
    during the verification to ensure agreement between the codes. Verification 
    of each code used the same problem description, the transition from 


    

    \section{Conclusions}

%\bibliographystyle{IEEEtran} 
\bibliography{bibliography}
   
\end{document}