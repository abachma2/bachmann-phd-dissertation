\documentclass{article}

\usepackage{multirow}
\usepackage{lineno}
\usepackage{xspace}
\usepackage{threeparttable}
\usepackage{caption}
\usepackage{subcaption}
\usepackage[hidelinks]{hyperref}
\usepackage[margin=1in]{geometry}
\modulolinenumbers[5]


%% `Elsevier LaTeX' style
\bibliographystyle{elsarticle-num}
%%%%%%%%%%%%%%%%%%%%%%%

%%%% packages and definitions (optional)
\usepackage{placeins}
\usepackage{booktabs} % nice rules (thick lines) for tables
\usepackage{microtype} % improves typography for PDF
\usepackage{hhline}
\usepackage{amsmath}

\usepackage{threeparttable, tablefootnote}

\usepackage{tabularx}


%% Special typesetting for Cyclus
\newcommand{\Cyclus}{\textsc{Cyclus}\xspace}%
\newcommand{\Cycamore}{\textsc{Cycamore}\xspace}%
\usepackage[acronym,shortcuts]{glossaries}
\newacronym[longplural={metric tons of heavy metal}]{MTHM}{MTHM}{metric ton of heavy metal}
\newacronym{ABM}{ABM}{agent-based modeling}
\newacronym{ACDIS}{ACDIS}{Program in Arms Control \& Domestic and International Security}
\newacronym{AHTR}{AHTR}{Advanced High Temperature Reactor}
\newacronym{ANDRA}{ANDRA}{Agence Nationale pour la gestion des D\'echets RAdioactifs, the French National Agency for Radioactive Waste Management}
\newacronym{ANL}{ANL}{Argonne National Laboratory}
\newacronym{API}{API}{application programming interface}
\newacronym{ARCH}{ARCH}{autoregressive conditional heteroskedastic}
\newacronym{ARE}{ARE}{Aircraft Reactor Experiment}
\newacronym{ARFC}{ARFC}{Advanced Reactors and Fuel Cycles}
\newacronym{ARMA}{ARMA}{autoregressive moving average}
\newacronym{ASME}{ASME}{American Society of Mechanical Engineers}
\newacronym{ATWS}{ATWS}{Anticipated Transient Without Scram}
\newacronym{BDBE}{BDBE}{Beyond Design Basis Event}
\newacronym{BIDS}{BIDS}{Berkeley Institute for Data Science}
\newacronym{BOL}{BOL}{Beginning-of-Life}
\newacronym{BSD}{BSD}{Berkeley Software Distribution}
\newacronym{CAFCA}{CAFCA}{ Code for Advanced Fuel Cycles Assessment }
\newacronym{CASL}{CASL}{Consortium for Advanced Simulation of Light Water Reactors}
\newacronym{CDTN}{CDTN}{Centro de Desenvolvimento da Tecnologia Nuclear}
\newacronym{CEA}{CEA}{Commissariat \`a l'\'Energie Atomique et aux \'Energies Alternatives}
\newacronym{CI}{CI}{continuous integration}
\newacronym{CNEC}{CNEC}{Consortium for Nonproliferation Enabling Capabilities}
\newacronym{CNEN}{CNEN}{Comiss\~{a}o Nacional de Energia Nuclear}
\newacronym{CNERG}{CNERG}{Computational Nuclear Engineering Research Group}
\newacronym{COSI}{COSI}{Commelini-Sicard}
\newacronym{COTS}{COTS}{commercial, off-the-shelf}
\newacronym{CSNF}{CSNF}{commercial spent nuclear fuel}
\newacronym{CTAH}{CTAHs}{Coiled Tube Air Heaters}
\newacronym{CUBIT}{CUBIT}{CUBIT Geometry and Mesh Generation Toolkit}
\newacronym{CURIE}{CURIE}{Centralized Used Fuel Resource for Information Exchange}
\newacronym{DAG}{DAG}{directed acyclic graph}
\newacronym{DANESS}{DANESS}{Dynamic Analysis of Nuclear Energy System Strategies}
\newacronym{DBE}{DBE}{Design Basis Event}
\newacronym{DESAE}{DESAE}{Dynamic Analysis of Nuclear Energy Systems Strategies}
\newacronym{DHS}{DHS}{Department of Homeland Security}
\newacronym{DOE}{DOE}{Department of Energy}
\newacronym{DOE-NE}{DOE-NE}{U.S. Department of Energy, Office of Nuclear Energy}
\newacronym{DRACS}{DRACS}{Direct Reactor Auxiliary Cooling System}
\newacronym{DRE}{DRE}{dynamic resource exchange}
\newacronym{DSNF}{DSNF}{DOE spent nuclear fuel}
\newacronym{DYMOND}{DYMOND}{Dynamic Model of Nuclear Development }
\newacronym{EBS}{EBS}{Engineered Barrier System}
\newacronym{EDZ}{EDZ}{Excavation Disturbed Zone}
\newacronym{EIA}{EIA}{U.S. Energy Information Administration}
\newacronym{EPA}{EPA}{Environmental Protection Agency}
\newacronym{EP}{EP}{Engineering Physics}
\newacronym{FCO}{FCO}{Fuel Cycle Options}
\newacronym{FCT}{FCT}{Fuel Cycle Technology}
\newacronym{FCWMD}{FCWMD}{Fuel Cycle and Waste Management Division}
\newacronym{FEHM}{FEHM}{Finite Element Heat and Mass Transfer}
\newacronym{FEPs}{FEPs}{Features, Events, and Processes}
\newacronym{FHR}{FHR}{Fluoride-Salt-Cooled High-Temperature Reactor}
\newacronym{FLiBe}{FLiBe}{Fluoride-Lithium-Beryllium}
\newacronym{GCAM}{GCAM}{Global Change Assessment Model}
\newacronym{GDSE}{GDSE}{Generic Disposal System Environment}
\newacronym{GDSM}{GDSM}{Generic Disposal System Model}
\newacronym{GENIUSv1}{GENIUSv1}{Global Evaluation of Nuclear Infrastructure Utilization Scenarios, Version 1}
\newacronym{GENIUSv2}{GENIUSv2}{Global Evaluation of Nuclear Infrastructure Utilization Scenarios, Version 2}
\newacronym{GENIUS}{GENIUS}{Global Evaluation of Nuclear Infrastructure Utilization Scenarios}
\newacronym{GPAM}{GPAM}{Generic Performance Assessment Model}
\newacronym{GRSAC}{GRSAC}{Graphite Reactor Severe Accident Code}
\newacronym{GUI}{GUI}{graphical user interface}
\newacronym{HALEU}{HALEU}{High Assay Low Enriched Uranium}
\newacronym{HEU}{HEU}{High Enriched Uranium}
\newacronym{HLW}{HLW}{high level waste}
\newacronym{HPC}{HPC}{high-performance computing}
\newacronym{HTC}{HTC}{high-throughput computing}
\newacronym{HTGR}{HTGR}{High Temperature Gas-Cooled Reactor}
\newacronym{IAEA}{IAEA}{International Atomic Energy Agency}
\newacronym{IEMA}{IEMA}{Illinois Emergency Mangament Agency}
\newacronym{INL}{INL}{Idaho National Laboratory}
\newacronym{IPRR1}{IRP-R1}{Instituto de Pesquisas Radioativas Reator 1}
\newacronym{IRP}{IRP}{Integrated Research Project}
\newacronym{ISFSI}{ISFSI}{Independent Spent Fuel Storage Installation}
\newacronym{ISRG}{ISRG}{Independent Student Research Group}
\newacronym{JFNK}{JFNK}{Jacobian-Free Newton Krylov}
\newacronym{LANL}{LANL}{Los Alamos National Laboratory}
\newacronym{LBNL}{LBNL}{Lawrence Berkeley National Laboratory}
\newacronym{LCOE}{LCOE}{levelized cost of electricity}
\newacronym{LDRD}{LDRD}{laboratory directed research and development}
\newacronym{LEU}{LEU}{Low Enriched Uranium}
\newacronym{LFR}{LFR}{Lead-Cooled Fast Reactor}
\newacronym{LGPL}{LGPL}{Lesser GNU Public License}
\newacronym{LLNL}{LLNL}{Lawrence Livermore National Laboratory}
\newacronym{LMFBR}{LMFBR}{Liquid-Metal-cooled Fast Breeder Reactor}
\newacronym{LOFC}{LOFC}{Loss of Forced Cooling}
\newacronym{LOHS}{LOHS}{Loss of Heat Sink}
\newacronym{LOLA}{LOLA}{Loss of Large Area}
\newacronym{LP}{LP}{linear program}
\newacronym{LWR}{LWR}{Light Water Reactor}
\newacronym{MARKAL}{MARKAL}{MARKet and ALlocation}
\newacronym{MA}{MA}{minor actinide}
\newacronym{MCNP}{MCNP}{Monte Carlo N-Particle code}
\newacronym{MILP}{MILP}{mixed-integer linear program}
\newacronym{MIT}{MIT}{the Massachusetts Institute of Technology}
\newacronym{MMR}{MMR}{Micro Modular Reactor}
\newacronym{MOAB}{MOAB}{Mesh-Oriented datABase}
\newacronym{MOOSE}{MOOSE}{Multiphysics Object-Oriented Simulation Environment}
\newacronym{MOX}{MOX}{mixed oxide}
\newacronym{MSBR}{MSBR}{Molten Salt Breeder Reactor}
\newacronym{MSRE}{MSRE}{Molten Salt Reactor Experiment}
\newacronym{MSR}{MSR}{Molten Salt Reactor}
\newacronym{NAGRA}{NAGRA}{National Cooperative for the Disposal of Radioactive Waste}
\newacronym{NCSA}{NCSA}{National Center for Supercomputing Applications}
\newacronym{NEAMS}{NEAMS}{Nuclear Engineering Advanced Modeling and Simulation}
\newacronym{NEUP}{NEUP}{Nuclear Energy University Programs}
\newacronym{NFCSim}{NFCSim}{Nuclear Fuel Cycle Simulator}
\newacronym{NFC}{NFC}{Nuclear Fuel Cycle}
\newacronym{NGNP}{NGNP}{Next Generation Nuclear Plant}
\newacronym{NMWPC}{NMWPC}{Nuclear MW Per Capita}
\newacronym{NNSA}{NNSA}{National Nuclear Security Administration}
\newacronym{NPRE}{NPRE}{Department of Nuclear, Plasma, and Radiological Engineering}
\newacronym{NQA1}{NQA-1}{Nuclear Quality Assurance - 1}
\newacronym{NRC}{NRC}{Nuclear Regulatory Commission}
\newacronym{NSF}{NSF}{National Science Foundation}
\newacronym{NSSC}{NSSC}{Nuclear Science and Security Consortium}
\newacronym{NUWASTE}{NUWASTE}{Nuclear Waste Assessment System for Technical Evaluation}
\newacronym{NWF}{NWF}{Nuclear Waste Fund}
\newacronym{NWTRB}{NWTRB}{Nuclear Waste Technical Review Board}
\newacronym{OCRWM}{OCRWM}{Office of Civilian Radioactive Waste Management}
\newacronym{ORION}{ORION}{ORION}
\newacronym{ORNL}{ORNL}{Oak Ridge National Laboratory}
\newacronym{PARCS}{PARCS}{Purdue Advanced Reactor Core Simulator}
\newacronym{PBAHTR}{PB-AHTR}{Pebble Bed Advanced High Temperature Reactor}
\newacronym{PBFHR}{PB-FHR}{Pebble-Bed Fluoride-Salt-Cooled High-Temperature Reactor}
\newacronym{PEI}{PEI}{Peak Environmental Impact}
\newacronym{PH}{PRONGHORN}{PRONGHORN}
\newacronym{PI}{PI}{Principal Investigator}
\newacronym{PNNL}{PNNL}{Pacific Northwest National Laboratory}
\newacronym{PRIS}{PRIS}{Power Reactor Information System}
\newacronym{PRKE}{PRKE}{Point Reactor Kinetics Equations}
\newacronym{PSPG}{PSPG}{Pressure-Stabilizing/Petrov-Galerkin}
\newacronym{PWAR}{PWAR}{Pratt and Whitney Aircraft Reactor}
\newacronym{PWR}{PWR}{Pressurized Water Reactor}
\newacronym{PyNE}{PyNE}{Python toolkit for Nuclear Engineering}
\newacronym{PyRK}{PyRK}{Python for Reactor Kinetics}
\newacronym{QA}{QA}{quality assurance}
\newacronym{RDD}{RD\&D}{Research Development and Demonstration}
\newacronym{RD}{R\&D}{Research and Development}
\newacronym{RELAP}{RELAP}{Reactor Excursion and Leak Analysis Program}
\newacronym{RIA}{RIA}{Reactivity Insertion Accident}
\newacronym{RIF}{RIF}{Region-Institution-Facility}
\newacronym{SAM}{SAM}{Simulation and Modeling}
\newacronym{SCF}{SCF}{Software Carpentry Foundation}
\newacronym{SFR}{SFR}{Sodium-Cooled Fast Reactor}
\newacronym{SINDAG}{SINDA{\textbackslash}G}{Systems Improved Numerical Differencing Analyzer $\backslash$ Gaski}
\newacronym{SKB}{SKB}{Svensk K\"{a}rnbr\"{a}nslehantering AB}
\newacronym{SNF}{SNF}{spent nuclear fuel}
\newacronym{SNL}{SNL}{Sandia National Laboratory}
\newacronym{SNM}{SNM}{Special Nuclear Material}
\newacronym{STC}{STC}{specific temperature change}
\newacronym{SUPG}{SUPG}{Streamline-Upwind/Petrov-Galerkin}
\newacronym{SWF}{SWF}{Separations and Waste Forms}
\newacronym{SWU}{SWU}{Separative Work Unit}
\newacronym{SandO}{S\&O}{Signatures and Observables}
\newacronym{THW}{THW}{The Hacker Within}
\newacronym{TRIGA}{TRIGA}{Training Research Isotope General Atomic}
\newacronym{TRISO}{TRISO}{Tristructural Isotropic}
\newacronym{TRU}{TRU}{transuranic}
\newacronym{TSM}{TSM}{Total System Model}
\newacronym{TSPA}{TSPA}{Total System Performance Assessment for the Yucca Mountain License Application}
\newacronym{UDB}{UDB}{Unified Database}
\newacronym{UFD}{UFD}{Used Fuel Disposition}
\newacronym{UML}{UML}{Unified Modeling Language}
\newacronym{UNFSTANDARDS}{UNFST\&DARDS}{Used Nuclear Fuel Storage, Transportation \& Disposal Analysis Resource and Data System}
\newacronym{USNC}{USNC}{Ultra Safe Nuclear Company}
\newacronym{UOX}{UOX}{uranium oxide}
\newacronym{UQ}{UQ}{uncertainty quantification}
\newacronym{US}{US}{United States}
\newacronym{UW}{UW}{University of Wisconsin}
\newacronym{VISION}{VISION}{the Verifiable Fuel Cycle Simulation Model}
\newacronym{VV}{V\&V}{verification and validation}
\newacronym{WIPP}{WIPP}{Waste Isolation Pilot Plant}
\newacronym{YMG}{YMG}{Young Members Group}
\newacronym{YMR}{YMR}{Yucca Mountain Repository Site}
\newacronym{NEI}{NEI}{Nuclear Energy Institute}
%\newacronym{<++>}{<++>}{<++>}
%\newacronym{<++>}{<++>}{<++>}


%\makeglossary

\begin{document}
\section{Introduction}
    Fuel cycle simulators have become an important tool in understanding 
    transition scenarios of the nuclear fuel cycle, or the transition 
    from one fuel cycle option to another. Common transition scenarios 
    modeled include the transition from a once-through fuel cycle with 
    \glspl{LWR} to recycling spent fuel with an advanced reactor. Important 
    information gained from fuel cycle simulations include material 
    flows and facility deployment schedules to meet an energy demand. 

    To accomodate the wide variety of transition scenarios and the multitude
    of parameters associated with fuel cycle modeling, multiple fuel 
    cycle simulators have been developed. Each simulator has been developed 
    to address a specific need, and most employ different types of modeling 
    frameworks. To help understand these differences, this work discusses 
    and compares some of the fuel cycle simulation 
    software that has been developed. The fuel cycle simulators 
    discussed in this work are not a comprehensive list of all available 
    simulators. 
    

\section{Review}
\subsection{DYMOND}
    \gls{DYMOND} was developed for the Generation IV Fuel Cycle Cross Cut 
    group and has been the primary 
    system dynamics model used by \gls{DOE} Advanced Fuel Cycle Initiative (AFCI)
    to evaluate fuel cycle scenarios of 
    advanced nuclear energy systems \cite{yacout_visionverifiable_2006}.
    It is run using the \textit{iThink} software with Microsoft Excel 
    templates as the software inputs and outputs \cite{feng_standardized_2016}.
    \gls{DYMOND} performs detailed system dynamic modeling for the whole nuclear 
    energy enterprise \cite{yacout_visionverifiable_2006}. This dynamic modeling includes 
    accounting for possible construction delays. 
    Inputs to DYMOND include reactor and fuel characteristics, fuel cycle
    facilities, possible material flow pathways, and the power demand of the 
    scenario \cite{feng_standardized_2016}. 
    \gls{DYMOND} utilizes recipes (composition vectors) to define reactor fuel 
    inputs and outputs. but current recipes available are limited 
    to specific reactor designs and burnups \cite{yacout_visionverifiable_2006}.
    If simulating a fuel cycle with recycling, the primary material flow control 
    is the availability of elemental plutonium.

    \gls{DYMOND} has been used for dynamic analysis of promising fuel cycles 
    identified by the AFCI \cite{yacout_dynamic_2000}. These simulations 
    showed how the waste needing a repository can be reduced by implementing 
    recycling, and that removing \gls{TRU} materials from the repository 
    can improve repository performance by reducing the short-term heat, 
    long-term heat, and long-term dose \cite{yacout_dynamic_2000}. This 
    analysis demonstrated how fuel cycle modeling can be used to inform  
    performance of future fuel cycle facilitites. 

\subsection{VISION}
    \Ac{VISION} was designed to address the AFCI objectives \cite{yacout_visionverifiable_2006}
    and calculates metrics to describe the characteristics and 
    consequences of the modeled fuel cycle \cite{yacout_visionverifiable_2006}. 
    These metrics are grouped by the \gls{DOE} AFCI program objectives and include 
    waste management, proliferation resistance, energy recovery, safety, 
    and economics \cite{yacout_visionverifiable_2006}. \gls{VISION} is designed 
    to simualte the entire fuel cycle, but it is not designed to manage the 
    fuel cycle, such as tracking individual fuel assemblies 
    \cite{yacout_visionverifiable_2006}.

    \gls{VISION} is comprised of multiple modules that evaluate each mass flow and 
    metric \cite{yacout_visionverifiable_2006}. \gls{VISION} primarily uses 
    recipes to define material compositions and does not perform
    neutronics calculations. However, it can use parameters output from external 
    neutronics calculations \cite{yacout_visionverifiable_2006}. To account 
    for changes in isotopic concentration (especially the short-lived 
    isotopes) \gls{VISION} includes isotopic decay modeling. This capability 
    assists in 
    understanding the waste forms in the fuel cycle \cite{yacout_visionverifiable_2006}.
    Similar to \gls{DYMOND}, \gls{VISION} models continuous fuel reprocessing at 
    every time step \cite{feng_standardized_2016}. 

    Potential uses of \gls{VISION} include investigating the timing issues 
    associated with reactor deployment or waste accumulation for reprocessing, 
    or evaluating the benefits of advanced fuel cycles to meet specific 
    criteria, such as the storage limits of a geologic repository 
    \cite{yacout_visionverifiable_2006}. One specific use of \gls{VISION}
    is the time-dependent analysis of transition scenarios, as shown in 
    \cite{piet_dynamic_2011}. This work 
    simulated the transition from LWRs to advanced reactors with different 
    forms of recycling. Time-dependent analysis provies understanding of the 
    true system evolution for a dynamic fuel cycle scenraio. This capability of 
    \gls{VISION} was used to evalauate methods to meet the ACFI strategic 
    objectives \cite{piet_dynamic_2011}. This work showed that uranium mass 
    dominates the waste produced from reactors, and thus dominates the focus
    to reduce waste production. This work also showed that the location and 
    mass-flow of weapons-usable \gls{TRU} material will depend on the 
    reactor technology deployed, but the plutonium quality 
    ($^{239}$Pu/$^{total}$PU) is not sensitive to the fuel cycle. These later
    results demonstrate how fuel cycle simulators can be used to evaluate 
    non-proliferation criteria. 

\subsection{Cyclus}
    \Cyclus was developed to address the lack of flexibility of other 
    simulators to use user-defined or custom technology and facilities. It 
    was also developed to improve comparison capabilities for simulations 
    eith and without these custom technologies \cite{huff_fundamental_2016}.

    \Cyclus is an open source agent-based fuel cycle simulator. Built 
    in C++, \Cyclus uses only open source and freely available libraries to 
    provide full access to all users and developers. The 
    \Cyclus architecture treats materials and facilities discretely and allows 
    for variable fidelity levels \cite{huff_fundamental_2016}. These attributes
    of \Cyclus allow for the software to easily model any fuel cycle scenario.

    \Cyclus uses the notion of an \textit{agent} to to represent different 
    components in the simulated fuel cycle. Agents are 
    defined using the \Cyclus application programming interface (API), which 
    defines and develops agents. Generic APIs are used to allow users 
    to develop their own suite of agent libraries and use them within \Cyclus. 
    The APIs anticipate the structure on information about a given library 
    that is required by the core \Cyclus kernel. This framework allows them 
    to facilitate 
    information sharing between the plug-in library and the \Cyclus framework. 
    In addition to the flexibility in libraries and agents, this framework 
    also allows for flexibility in licensing and distribution of the user 
    defined libraries \cite{huff_fundamental_2016}.

    The agent-based modeling paradigm employed by \Cyclus allows agent level 
    modeling, as opposed to system level modeling. This modeling paradigm 
    allows difference 
    fuel cycle facilities, such as a reactor and a fuel fabrication plant, to 
    be defined independently but still interact with each other in the 
    simulation. There are three main groups of agents within the \Cyclus 
    architecture: facilities, institutions, and regions. Facilities are 
    the individual units in the fuel cycle that implement technology, 
    such as a fuel fabrication facility or a uranium mine. Institutions 
    manage the facilities, similar to a company. Regions provide geographic 
    and political context for the institutions and regions, and can be thought 
    of as similar to individual nations. Each type of agent has its own 
    class wihtin \Cyclus. 

    The types of agents are related using a parent-child hierarchy: regions are
    parents of institutions, and institutions are parents of facilities. This 
    structure requires that institutions are responsible for the deployment 
    and decommissioning of the facilities. It also allows for advanced logic 
    to be implemented wth respect to facility building and decommissioning, 
    such as preferential regional or institutional trading (e.g. tariffs or 
    contracts). 

    Mutliple additions and extensions to \Cyclus have been developed, including 
    a way to couple \Cyclus to ORIGEN, a fuel depletion software, to increase
    \Cyclus's capability to perform dynamic, physics-based fuel depletion 
    within a fuel cycle scenario \cite{skutnik_cyborg:_2016}. Another 
    addition to \Cyclus is the automatic deployment of multiple fuel 
    cycle facilities (not just reactors) to meet a defined power demand 
    for the modeled scenario \cite{chee_demand-driven_2020}.    

\subsection{ORION}
    ORION is a fuel cycle simulator developed by the UK National Nuclear 
    Laboratory to track material movement through a user-defined nuclear 
    fuel cycle, model decay of radionuclides, and model transmutation of fuel
    \cite{gregg_analysis_2012}. The modeling is accurate enough to provide 
    high-fidelity radioactive decay and tracking of over 2000 radionuclides
    \cite{feng_standardized_2016}. 
    
    ORION has the ability to model each facility 
    as individual objects, which can be grouped together to represent small-scale 
    facilities. It can also automate facility deployment to meet electricity demand
    \cite{feng_standardized_2016}, simialar to what has been developed through 
    the \Cycamore archetypes for \Cyclus \cite{scopatz_cyclus_2015} and 
    the demand-driven deployment advances in \Cyclus \cite{chee_demand-driven_2020}.  
    
    One distinuguishing feature of ORION is that 
    it can simulate in-core irradiation of fuel, using externally 
    generated lattice physics calculations prior to the ORION analysis
    \cite{feng_standardized_2016}. This feature is different from the standard 
    capabilities of \Cyclus, \gls{DYMOND}, and \gls{VISION}, which all use defined 
    recipes for fuel before and after the reactor. Generating cross section 
    data and performing neutronics calculations is most similar to 
    the CyBORG archetype in \Cyclus \cite{skutnik_cyborg:_2016}. Another 
    distinuguishing feature is that ORION explicitly models the timing of 
    each fuel reload between cycle lengths \cite{feng_standardized_2016}.
    This feature is contrasted with \gls{DYMOND} and \gls{VISION} which both average
    the mass flow over a time step \cite{feng_standardized_2016}.

    One application of ORION was to evaluate the transition from the current fleet of 
    Light Water Reactors in the U.S. to a fleet of sodium fast reactors with 
    continuius recycling \cite{sunny_transition_2015}. This analysis coupled 
    ORION with SCALE to create problem specific cross sections, expanding the 
    capabilities of ORION to ensure fidelity between reactor design and 
    the fuel cycle simulation \cite{sunny_transition_2015}. 
    
\subsection{Comparison in performance}
    There was a multi-lab effort to compare the above mentioned fuel cycle 
    simulators. Each national lab in the project used a different 
    fuel cycle simulator: \gls{ORNL} used ORION, \gls{INL} used 
    \gls{VISION}, \gls{ANL} used \gls{DYMOND}, and \gls{LLNL} used
    \Cyclus. Each lab simulated the same transition from a once-through 
    fuel cycle to one with continuous recyling of uranium and plutonium 
    with natural uranium in a fleet of critial fast reactors. This 
    fuel cycle option was identified as a scenario of interest in the 
    \gls{DOE-NE} Evaluation \& Screening Report \cite{wigeland_nuclear_2014}.
    
    The analysis using \Cyclus used a fixed reactor deployment schedule established 
    using \gls{DYMOND}, and the results qualitatively match the results from 
    \gls{DYMOND} despite the differences in their modeling strategies 
    \cite{djokic_application_2015}. Despite these promising results the 
    reactor deployment schedule should be improved to better match the 1\% 
    annual increase in energy demand in the \Cyclus simulation
    \cite{djokic_application_2015}. The results of the other parts of 
    this project could not be found via open-source resources. 


\subsubsection{Verification efforts}
    Another effort to compare fuel cycle simulators was in their verification
    \cite{feng_standardized_2016,bae_standardized_2019}. Verification efforts
    compared the performance of DYMOND, VISION, ORION, MARKAL (not included in 
    this work)
    \cite{feng_standardized_2016} and \Cyclus later \cite{bae_standardized_2019}, 
    making adjustments to the codes as needed
    during the verification to ensure agreement between the codes. Verification 
    of each code used the same problem description: the transition from 
    a fleet of 100 \glspl{LWR} to \glspl{SFR} with reprocessing assuming a
    0\% energy demand growth.  
    The scenario was designed to be simple enough to allow for a comparison 
    to be made with results from a spreadsheet using basic arithmetic. The 
    results of this 
    verification showed that each of the codes and the spreadsheet were in 
    good agreement, with the exception of a few differences resulting from 
    variations in the code functions \cite{feng_standardized_2016}.

    The verification of \Cyclus identified two primary differences between 
    this code and the others verified \cite{bae_standardized_2019}. The 
    first is the core depletion upon reactor decommissioning. The \Cycamore
    reactor archetype deplets half of the core upon decommissioning, while the 
    other codes deplete the entire core. This difference was mitigated by 
    modifying the \Cycamore source code to match the other codes. The other 
    difference identified is that \Cyclus and ORION both fully resolve 
    discrete batches of fuel discharge, while the other codes do not. This 
    difference stems from modeling choices made in each of the codes 

    

\section{Conclusions}
    Multiple fuel cycle simulators have been developed to address a variety 
    of needs within the frame of fuel cycle modeling. Fuel cycle simulators 
    can be used for a wide range of purposes, including evaluating transition
    scenarios and evaluate non-proliferation criteria. Some fuel cycle 
    simulators have been developed to address specific objectives, while others 
    have been developed to increase the flexibility of fuel cycle modeling.
    The simulators also have a wide variety of capabilities and modeling frameworks.
    Some focus on dynamic scenerios, while others focus on static scenarios. 
    One common feature among fuel cycle simulators is the use of recipes to 
    define material compositions, although some (such as ORION and the CyBORG 
    archetype of \Cyclus) allow for depletion or irradiation calculations 
    to determine specific spent fuel compositions. One area of need in the 
    further development of these fuel cycle simulators is the expansion of their 
    energy technologies. They currently are only able to simulate transitions 
    within the scope of nuclear reactor technology. A beneficial expansion 
    would allow for their use in the evaluation of transitions between energy 
    sources, such as transitioning from natural gas to nuclear reactors. 
%\bibliographystyle{IEEEtran} 
\bibliography{bibliography}
   
\end{document}