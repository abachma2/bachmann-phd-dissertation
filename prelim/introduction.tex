\section{Motivation}
The current nucler fuel cycle in the US is based on supplying reactors 
with fuel comprised of enriched uranium. Natural uranium consists of 
about 0.7\% uranium-235 by weight, with most of the rest of the weight being 
uranium-238. Enriched uranium sent to reactors is enriched up to 4.95\% 
uranium-235 \hl{Find citation, NRC regs}. The US nuclear industry is 
interested in developing a supply chain for \gls{HALEU}, which is 
uranium enriched to between 5-20\% uranium-235 \hl{Find citation}. \gls{HALEU}
has a variety of potential uses, including for medical isotope production,
test reactors, and thermal propulsion for space exploration \cite{nagley_ha-leu_2020}.
For nuclear power reactors, \gls{HALEU} fuel is expected to provide benefts 
over current fuel enrichment levels, such as higher burnups \hl{citation},
longer cycle times, and 
reducing the \gls{LCOE} for the reactors \cite{carlson_economic_2020}.

Two methods can be used to produce \gls{HALEU}: enrich natural uranium to the 
required level or downblend \gls{HEU} to the required level. The current US
nuclear commercial fuel cycle does not have the capability to produce 
\gls{HALEU} using either of these methods \hl{citation}. Enriching uranium 
to the required enrichment level is limited by the amount of natural uranium 
that can be mined and the \gls{SWU} capacity of the available facilities. 
Downblending \gls{HEU} is limited by the amount of \gls{HEU} avaialable and 
the downblending capacity of the available facilities. 

There is currently not a commercial facility in the US that can enrich 
uranium to produce \gls{HALEU} \cite{hussain_nei_2018}, which has led 
many to consider downblending \gls{HEU} to obtain an initial stockpile of 
\gls{HALEU}. 
Currently, there is only one facility in the US commercially licensed to 
downblend \gls{HEU}, the BWXT Nuclear Fuel Services Inc. facility in 
Erwin, TN, which is expected to have the capacity to downblend 1-2 
MT of \gls{HEU} into up to 10 MT of \gls{HALEU} each year \cite{nagley_ha-leu_2020}.
The \gls{HEU} being considered for downblending mostly comes from the spent 
fuel of the \hl{EBR-II} reactor at \gls{INL} \cite{patterson_haleu_2019}. From 
this stockpile, there is about three metric tonnes of \gls{HEU} that 
could be downblended \cite{patterson_haleu_2019}.



\section{Research Goals}
The goal of this work is to investigate the impacts of deploying reactors fueled by 
\gls{HALEU} in the United States, including the impacts on the nuclear fuel cycle and 
on the reactors deploy. This goal is accomplished 
through multiple steps. The first step is modeling of multiple transition scenarios
to select advanced reactors requiring \gls{HALEU} fuel. The next is performing 
sensitivity analysis on the modeled transition scenarios and optimize the scenarios
based on perscribed metrics. Finally, the impact of the \gls{HALEU} production 
methods on reactor performance is investigated. 

The structure of this thesis is as follows. Chapter 2 provides some 
background information on the nuclear fuel cycle, nuclear fuel cycle 
modeling efforts, and the use of \gls{HALEU}-fuel in reactors.
Chapter 3 discusses the methodology for each section of this work, 
including the fuel cycle scenarios modeled, the advanced reactor selection 
method, and the neutronics models. Chapter 4 discusses the results of the 
once-through fuel cycle transition scenarios. Chapter 5 discusses the 
results of the recycling transition scenarios. Chapter 6 discusses the results 
of the sensitivity analysis and the optimization schemes applied to the 
transition scenarios. Chapter 7 discusses the impacts of various \gls{HALEU}
production methods on the performance of the reactors. Finally, Chapter 
8 provides some concluding remarks and suggestions for future work. 