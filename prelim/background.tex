\section{The nuclear fuel cycle}
The nuclear fuel cyclye encompasses the activities and processes 
for the use of fissile materials in fission nuclear reactors \cite{tsoulfanidis_nuclear_2013}. 
This can include the use of uranuim, thorium, or even plutonium as the 
fuel for the reactor, with uranium being the most widely used fuel for 
power reactors \hl{FIND CITATION}. This work focuses on a uranium based 
fuel cycle. The nuclear fuel cycle begins with the mining of uranium ores from the earth 
and ends with either the final disposal of the radioactive waste produced 
\cite{tsoulfanidis_nuclear_2013}. There are mutliple variations of 
the uranium-based fuel cycle that differ based on the type and design of 
the reactors deployed. Two primary variations on the nuclear fuel cycle 
include the once-through and uranium and plutonium recycling fuel cycles 
\cite{tsoulfanidis_nuclear_2013}.

\subsection{Once-through fuel cycle}
A once-through fuel cycle is characterized by fuel only going through one 
pass through a reactor before being disposed of (Figure \ref{fig:once_through_background}).
\begin{figure}
    \centering
    \begin{tikzpicture}[node distance=1.5cm]
        \node (ore) [facility] {Uranium ore};
        \node (enrichment) [facility, right of=ore, xshift=1cm]{Enrich};
        \node (llwsink) [facility, below of=enrichment]{Depleted uranium};
        \node (fabrication) [facility, right of=enrichment, xshift=1cm]{Fuel};
        \node (reactor) [facility, above of=fabrication, xshift=2cm]{Power Plant};
        \node (spentfuel) [facility, below of=reactor, xshift=2cm]{Interim Storage};
        \node (sinkhlw) [facility, below of=spentfuel, xshift=2cm]{Geologic Repository};

        \draw [arrow] (ore) --  (enrichment); 
        \draw [arrow] (enrichment) -- (fabrication);
        \draw [arrow] (enrichment) -- (llwsink);
        \draw [arrow] (fabrication) |- (reactor);
        \draw [arrow] (reactor) -| (spentfuel);
        \draw [arrow] (spentfuel) -| (sinkhlw);

        \end{tikzpicture}
    \caption{Material flow for a once-through nuclear fuel cycle. All material is 
    disposed of after use in a reactor. Adapted from \protect\cite{wigeland_identification_2011}.}
    \label{fig:once_through_background}
\end{figure}

\subsection{Recycling fuel cycle}
A recycling fuel cycle is characterized by fuel going through additional 
passes through a reactor after a separtations step to remove fission 
products (Figure \ref{fig:recycle_background})
\begin{figure} 
    \centering
    \begin{tikzpicture}[node distance=1.5cm]
        \node (ore) [facility] {Uranium ore};
        \node (enrichment) [facility, right of=ore, xshift=1cm]{Enrich};
        \node (llwsink) [facility, below of=enrichment]{Depleted uranium};
        \node (fabrication) [facility, right of=enrichment, xshift=1cm]{Fuel};
        \node (reactor) [facility, above of=fabrication, xshift=2cm]{Power Plant};
        \node (spentfuel) [facility, below of=reactor, xshift=2cm]{Interim Storage};
        \node (sinkhlw) [facility, below of=spentfuel, xshift=4cm]{Geologic Repository};
        \node (separation) [facility, below of=reactor, yshift=-1.5cm]{Separation};

        \draw [arrow] (ore) --  (enrichment); 
        \draw [arrow] (enrichment) -- (fabrication);
        \draw [arrow] (enrichment) -- (llwsink);
        \draw [arrow] (fabrication) |- (reactor);
        \draw [arrow] (reactor) -| (spentfuel);
        \draw [arrow] (spentfuel) -| node[anchor=west]{spent fuel}(sinkhlw);
        \draw [arrow] (spentfuel) |- (separation);
        \draw [arrow] (separation) -| (fabrication);
        \draw [arrow] (9,-1.0)-- node[anchor=north, text width=1.5cm]{fission products}(sinkhlw);
        
        \end{tikzpicture}
    \caption{Material flow for a once-through nuclear fuel cycle. All material is 
    disposed of after use in a reactor. Adapted from \protect\cite{wigeland_identification_2011}.}
    \label{fig:recycle_background}
\end{figure}

\subsection{Enrichment}
In the nuclear fuel cycle, enrichment increases the relative abundance of 
uranium-235 compared to uranium-238 in the fuel. The capacity of an 
enrichment facility is based on a number of quantities \cite{tsoulfanidis_nuclear_2013}:
\begin{subequations}
    \begin{equation}
        F = \text{mass (kg) of feed material per unit time}
    \end{equation}
    \begin{equation}
        P = \text{mass (kg) of product material per unit time}
    \end{equation}
    \begin{equation}
        T = \text{mass (kg) of tails produced per unit time}
    \end{equation}
    \begin{equation}
        x_f = \text{weight fraction of $^{235}$U in the feed stream}
    \end{equation}
    \begin{equation}
        x_p = \text{weight fraction of $^{235}$U in the product stream}
    \end{equation}
    \begin{equation}
        x_t = \text{weight fraction of $^{235}$U in the tails stream}
    \end{equation}
    \begin{equation}
        SWU = \text{separtive work units (SWU); the physical work required to separate the isotopes}
    \end{equation}
\end{subequations}

Each of these quantites can be related using the following equations:
\begin{subequations}
    \begin{equation}
        F = P + T
    \end{equation}
    \begin{equation}
        x_fF = x_pP + x_tT
    \end{equation}
    \label{eq:enrichment_2}
    \begin{equation}
        SWU = [P*V(x_p) +T*V(x_t) - F*V(x_f)]*t
    \end{equation}
    \text{in which:}
    \begin{equation}
        V(x_i) = (2x_i - 1)*\ln\left(\frac{x_i}{1-x_i}\right)
    \end{equation}
    \begin{equation}
        t = \text{a given time period}
    \end{equation}
\end{subequations}

\section{Fuel cycle simulators}
Fuel cycle simulators are computational tools to model the flow of materials
and the commissioning and decommissioning of facilities for a given fuel 
cycle. Simulators can be used to evaluate the transition between fuel cycle 
options (e.g. from a once-through fuel cycle to one with recycling), the 
performance of one fuel cycle with a growth in the demand of nuclear power, 
and the effect of perturbations on a given fuel cycle \cite{piet_dynamic_2011}. 

A variety of fuel cycle simulators have been developed to meet 
different needs, including VISION \cite{yacout_visionverifiable_2006}, 
DYMOND \cite{yacout_visionverifiable_2006}, and ORION \cite{gregg_analysis_2012}. 

\subsection{Uses of fuel cycle simulators}
VISION was used to evaluate 

\subsection{\Cyclus}
\Cyclus is a dynamic, open source agent-based fuel cycle simulator. Built 
in C++, \Cyclus uses only open source and freely available libraries to 
provide full access to all users and developers. The 
\Cyclus architecture treats materials and facilities discretely and allows 
for variable fidelity levels \cite{huff_fundamental_2016}. These attributes
of \Cyclus allow for the software to easily model any fuel cycle scenario.

\Cyclus uses the notion of an \textit{agent} to to represent different 
components in the simulated fuel cycle. Agents are 
defined using the \Cyclus application programming interface (API), which 
defines and develops agents. Generic APIs are used to allow users 
to develop their own suite of agent libraries and use them within \Cyclus. 
The APIs anticipate the structure on information about a given library 
that is required by the core \Cyclus kernel. This allows them to facilitate 
information sharing between the plug-in library and the \Cyclus framework. 
In addition to the flexibility in libraries and agents, this framework 
also allows for flexibility in licensing and distribution of the user 
defined libraries. Libraries are loaded without changes to the \Cyclus 
kernel and without unwanted transfer of sensitive information. 

The agent-based modeling paradigm employed by \Cyclus allows agent level 
modeling, as opposed to system level modeling. This allows difference 
fuel cycle facilities, such as a reactor and a fuel fabrication plant, to 
be defined independently but still interact with each other in the 
simulation. There are three main groups of agents within the \Cyclus 
architecture: facilities, institutions, and regions. Facilities are 
the individual units in the fuel cycle that implement technology, 
such as a fuel fabrication facility or a uranium mine. Institutions 
manage the facilities, similar to a company. Regions provide geographic 
and political context for the institutions and regions, and can be thought 
of as similar to individual nations. Each type of agent has its own 
class wihtin \Cyclus. 

The types of agents are related using a parent-child hierarchy: regions are
parents of institutions, and institutions are parents of facilities. This 
structure requires that institutions are responsible for the deployment 
and decommissioning of the facilities. It also allows for advanced logic 
to be implemented wth respect to facility building and decommissioning, 
such as preferential regional or institutional trading (e.g. tariffs or 
contracts). 

\subsection{Verification and Use}
Verification of \Cyclus has been performed \cite{bae_standardized_2019}, 
based on a transition scenario from an open fuel cycle to an advanced
fuel cycle with reprocessing \cite{feng_standardized_2016}. Due to 
some inherent differences in \Cyclus and the other fuel cycle simulators 
used by \cite{feng_standardized_2016} there were some differences in 
the scenario parameters, such as \gls{LWR} batch size or 
cycle length. These differences are implemented to conserve the reactor 
core size, with a negligible difference in the \gls{SFR} core size. \Cyclus 
was shown to have strong agreement with the verification of other 
fuel cycle simulators \cite{bae_standardized_2019} with respect to 
multiple metrics such as reactor startup and decommission schedule, fresh 
fuel loading, and annual reprocessing throughputs. One major difference
between \Cyclus and other fuel cycle simulators identified by this work 
is that \Cyclus fully resolves discrete batches for fuel discharge. This 
implementation allows \Cyclus to better match realistic fuel handeling, but 
does not prevent \Cyclus from producing comparable results to other 
fuel cycle simulators. 

\Cyclus has been used to model the EG 23 transition scenario 
\cite{wigeland_nuclear_2014}, assuming a 1\% growth in power demand 
\cite{djokic_application_2015}. The results from \Cyclus were compared 
to the results from modeling the same transition scenario with DYMOND, 
a fuel cycle simulator from \gls{ANL}. The power generated, reactor entry 
time, and reactor exit time for the \Cyclus simulation qualitatively 
matches well with the results of the DYMOND simulation, but the power 
demand growth could match better to the target 1\% by slightly modifying 
the \gls{SFR} deployment. 

\section{Fuel cycle sensitivity analysis}

\section{HALEU}
\gls{HALEU} is defined as uranium that is enriched between 5-20\% by mass in 
uranium-235, and is considered a subset of \gls{LEU}. Fuel for \glspl{LWR} 
is enriched to no more than 5\% uranium-235, per \gls{NRC} regulations 
\hl{citation}. Table \ref{tab:enrichemnt} provides definitions of some of the 
main classifications of uranium enrichemnt. 

\begin{table}
    \centering
    \caption{Categories of uranium enrichment by weghit fraction of 
    uranium-235.}
    \label{tab:enrichemnt}
    \begin{tabular}{l c c}
        \hline
        Category & Weight fraction (\%)\\\hline
        Depleted & <0.711 \\
        Natural & 0.711 \\
        \gls{LEU} & 0.711-20 \\
        \gls{HALEU} & 5-20 \\
        \gls{HEU} & $\ge$20 \\
        \hline
    \end{tabular}
\end{table}

\subsection{Production Methods}
There are two primary methods of producing \gls{HALEU}. The first 
option is to enrich uranium to the required level. There is only one 
facility in the US currently 
licensed to enrich uranium, the Urenco LES facility in Eunice, 
New Mexico \cite{noauthor_establishing_2022}. This facility is only 
licensed to enrich uranium up to 5.5\% uranium-235 \cite{noauthor_establishing_2022},
which is less than many of the \gls{HALEU}-fueled reactor designs 
require. Another enrichment facility is currently under development in 
Piketon, Ohio which is licensed to enrich uranium up to 20\% 
uranium-235 \cite{noauthor_establishing_2022}. However, this is only 
designed to be a demonstration facility that produces less than 1 MT of 
\gls{HALEU} \cite{noauthor_establishing_2022} with the possibility of 
developing more facilities to produce up to 12 MTU/year of \gls{HALEU}
\cite{noauthor_establishing_2022}.

The other option is to downblend \gls{HEU} to the required \gls{HALEU}
level. The \gls{DOE} is considering muliple sources of \gls{HEU} that can 
be used to create \gls{HALEU} \cite{noauthor_establishing_2022}. The 
first is spent fuel from \gls{EBR}, which can produce a total of 10 MTU 
of \gls{HALEU} at 19.75\% enrichment \cite{noauthor_establishing_2022}. 
The next source is the stored \gls{HEU} solution at \gls{SRS} from 
spent research reactor fuel. This source is smaller than what is 
available from \gls{EBR}, and would only produce 2 MTU of \gls{HALEU} 
at 19.75\% enrichment \cite{noauthor_establishing_2022}. These sources of 
\gls{HALEU} would certainly be beneficial to fueling reactors, but they 
are a very limited supply of \gls{HALEU} that can not be counted on 
for long-term support of \gls{HALEU} reactors. 


\subsection{Fuel forms}
\gls{HALEU} can be used in a variety of fuel forms. One of the possible 
forms is as \gls{TRISO} fuel particles. \gls{TRISO} fuel particles contain 
a small fuel kernel surrounded by four layers of material: a porous caron 
buffer layer, an inner pyrolytic carbon layer, a silicon carbide layer, and 
an outer pyrolytic carbide layer \cite{powers_fully_2014}. The carbon buffer
layer helps to accomodate swelling or shrinking of the fuel kernel and 
accumulate internal gases released from the kernel. The inner and outer 
pyrolytic carbon layers protect the silicon carbide layer from chemical 
reactions and help prevent the diffusion of any fission products out of the 
particle. The silicon carbide layer helps prevent the diffusion of fission 
products out of the particle. The total diameter for one \gls{TRISO} 
particle is on the order of 800-900 $\mu$m \hl{citation}. \gls{TRISO} fuel 
particles are often used in 
larger spherical pebbles or hexagonal graphite prismatic blocks 
\cite{powers_fully_2014}.

\gls{TRISO} particles with \gls{HALEU} are used in a variety of reactor designs 
\cite{harlan_x-energy_2018,mitchell_usnc_2020,hussain_advances_2018}.

\subsection{Expected Demand}
A survey of select advanced reactor companies by \gls{NEI} reports an 
expected cumulative demand of \gls{HALEU} of almost 600 MT by 2030 
\cite{korsnick_need_2018}.

Demand for ARDP reactors will begin in 2024, so that fuel is 
properly fabricated in time for reactors to come on-line \cite{noauthor_establishing_2022}.
Other companies are expecting to deploy working \gls{HALEU}-fueled reactors 
by the mid-2020's \cite{noauthor_establishing_2022}.

An estimated 20 MTU of \gls{HALEU} will be needed between 2024-2027 to 
meet the needs of already announced projects in the US with a site 
selected \cite{noauthor_establishing_2022}. An estimated 6 MTU/year of 
\gls{HALEU} will be needed in 2028 to support refueling of these projects 
\cite{noauthor_establishing_2022}. 

\subsection{HALEU in reactors}
A lot of work has been done to evaluate the performance of reactors 
using \gls{HALEU} fuel. Burns et al. investigated the reactor and fuel cycle 
performance of an \gls{LWR} fueled by an enrichment above 5\% \cite{burns_reactor_2020}.
Their work showed that increasing the fuel enrichment up to 7\% does not 
largely affect the fuel temperature or the moderator temperature coefficents,
but it does decrease the soluble boron coefficient and increase the maximum 
burnup ar the edges of the fuel pellets. The impacts on the fuel cycle include 
an increase in the amount of natural uranium required, a decrease in the 
amount of high-level waste disposed of per unit energy, and changes in the 
waste radioacitivity waste. The increase in natural uranium matches 
expectations
because the product mass (i.e. mass of fuel in the core) per unit 
energy is the same, and 
based on \ref{eq:enrichment_2} the increase in the product enrichment would 
lead to an increase in the feed mass. 

Increasing the fuel enrichment of the NuScale \gls{SMR} has also been investigated 
\cite{carlson_implications_2022}. Increasing the fuel enricment to 8.34\% doubled the 
fuel discharge burnup and the cycle time. Fission product poisons, specifically 
sumarium-149 and xenon-135, increased on concentration when \gls{HALEU} fuel was used. 
This increase impacts the neutrons poisons that are needed during operation and the 
radioactivity of the fuel after discharge. Using \gls{HALEU} in the core also led 
to a reduction in the \gls{LCOE} for the reactor for certain combinations of 
enrichment and cycle length \cite{carlson_economic_2020,carlson_implications_2022}.

If \gls{HALEU} fuel is produced by downblending \gls{HEU}, is it very possible that 
the \gls{HALEU} will contain impurities that would not be present if natural uranium 
were enriched to produce the \gls{HALEU} \cite{noauthor_establishing_2022}. Bounding 
studies on the uranium isotopic 
composition of \gls{HALEU} produced from spent fuel from \gls{EBR} shows the presence 
of uranium-232, 233, 234, and 236 \cite{vaden_isotopic_2018}. \gls{HALEU} created from 
downblened \gls{HEU} at the Y-12 Nuclear Security Complex also has uranium-232, 234, and 
236 impurities \cite{nelson_foreign_2010}. Uranium-233 is a fissile 
isotope and is expected to affect the neutron population inside a reactor. Uranium-232, 
234, and 236 are parasitic neutron absorbers, so their presence is also expected to affect 
the neutron population. The neutronics performance of using \gls{HALEU} containing
uranium-235 and 238 was compared with the performance using \gls{HALEU} from Y-12 with the 
known impurities \cite{celikten_effects_2021}. Using 
the \gls{HALEU} with the \gls{HEU} impurities showed a 0.43\% reactivity drop, leading to 
an 8\% reduction in the cycle length. 
