\section{Transition Scenarios}


\section{Reactors}

\section{\Cyclus}
\Cyclus is a dynamic, open source agent-based fuel cycle simulator. Built 
in C++, \Cyclus uses only open source and freely available libraries to 
provide full access to all users and developers. The 
\Cyclus architecture treats materials and facilities discretely and allows 
for variable fidelity levels \cite{huff_fundamental_2016}. These attributes
of \Cyclus allow for the software to easily model any fuel cycle scenario.

\Cyclus uses the notion of an \textit{agent} to to represent different 
components in the simulated fuel cycle. Agents are 
defined using the \Cyclus application programming interface (API), which 
defines and develops agents. Generic APIs are used to allow users 
to develop their own suite of agent libraries and use them within \Cyclus. 
The APIs anticipate the structure on information about a given library 
that is required by the core \Cyclus kernel. This allows them to facilitate 
information sharing between the plug-in library and the \Cyclus framework. 
In addition to the flexibility in libraries and agents, this framework 
also allows for flexibility in licensing and distribution of the user 
defined libraries. Libraries are loaded without changes to the \Cyclus 
kernel and without unwanted transfer of sensitive information. 

The agent-based modeling paradigm employed by \Cyclus allows agent level 
modeling, as opposed to system level modeling. This allows difference 
fuel cycle facilities, such as a reactor and a fuel fabrication plant, to 
be defined independently but still interact with each other in the 
simulation. There are three main groups of agents within the \Cyclus 
architecture: facilities, institutions, and regions. Facilities are 
the individual units in the fuel cycle that implement technology, 
such as a fuel fabrication facility or a uranium mine. Institutions 
manage the facilities, similar to a company. Regions provide geographic 
and political context for the institutions and regions, and can be thought 
of as similar to individual nations. Each type of agent has its own 
class wihtin \Cyclus. 

The types of agents are related using a parent-child hierarchy: regions are
parents of institutions, and institutions are parents of facilities. This 
structure requires that institutions are responsible for the deployment 
and decommissioning of the facilities. It also allows for advanced logic 
to be implemented wth respect to facility building and decommissioning, 
such as preferential regional or institutional trading (e.g. tariffs or 
contracts). 

\subsection{Verification and Use}
Verification of \Cyclus has been performed \cite{bae_standardized_2019}, 
based on a transition scenario from an open fuel cycle to an advanced
fuel cycle with reprocessing \cite{feng_standardized_2016}. Due to 
some inherent differences in \Cyclus and the other fuel cycle simulators 
used by \cite{feng_standardized_2016} there were some differences in 
the scenario parameters, such as \gls{LWR} batch size or 
cycle length. These differences preserve the reactor core size, with a 
negligible difference in the \gls{SFR} core size. 