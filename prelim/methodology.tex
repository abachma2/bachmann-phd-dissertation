\section{Fuel cycle models}
The fuel cycle scenarios created show the resources required for the transition from the 
current fleet of \glspl{LWR} in the US to advanced reactors using \gls{HALEU}.
All fuel cycle models were run in \Cyclus 
\cite{huff_fundamental_2016}. Each simulation models current 
\glspl{LWR} starting in 1965 and models all reactors out to 2090. Start and 
end dates for each \gls{LWR} were obtained from the \gls{IAEA} \gls{PRIS} database 
\cite{noauthor_power_1989}. Reactors still in operation in December 2020, and thus 
lacking an end date in the \gls{PRIS} database are assumed to operate through their 
current operating license 
\cite{us_nuclear_regulatory_commission_clinton_2021,us_nuclear_regulatory_commission_comanche_2021,us_nuclear_regulatory_commission_comanche_2021-1,us_nuclear_regulatory_commission_perry_2021,us_nuclear_regulatory_commission_watts_2021,us_nuclear_regulatory_commission_watts_2021-1,nuclear_energy_institute_initial_2021,nuclear_energy_institute_second_2021}
Only reactors with a power level above 400 MWe were used in the simulation 
to avoid including prototype and research reactors present in the database. 
Approximate masses for fuel used in the core the \glspl{LWR} was obtained 
from \cite{todreas_nuclear_2012} and \cite{cacuci_handbook_2010}. 

Material compositions for each commodity in the models are defined using recipes.
Recipes for \gls{LWR} fresh and spent fuel were found in \cite{yacout_visionverifiable_2006}.
Obtaining recipes for fresh and spent fuel for the advanced reactors is described in 
Section \ref{sec:reactor_methods}.

Fuel cycle scenarios are grouped into once-through fuel cycles and recycle fuel 
cycles. Within each group, fuel cycle models are defined based on the energy demand 
of the scenario (either no growth or 1\% annual growth in demand) and by the 
advanced reactors deployed in the scenario. The transition from \glspl{LWR} to 
advanced reactors is modeled as beginning in 2025, so the energy demand is based on the 
energy supplied by \glspl{LWR} in 2025. Each of the fuel cycle scenarios 
defined in this work are compared on the number of reactors deployed, the 
uranium usage (both uranium ore and what is sent to the reactors), the enrichment 
\gls{SWU} capacity required, and the amount of waste produced. 

\subsection{Reactors} \label{sec:reactor_methods}
This work considers three advanced reactors: the \gls{USNC} \gls{MMR}, 
\cite{mitchell_usnc_2020}, the X-energy Xe-100 
\cite{harlan_x-energy_2018,hussain_advances_2018}, and the NuScale VOYGR reactor
\cite{nuscale_chapter_2020,nuscale_chapter_2020-1}. The \gls{USNC} \gls{MMR}
and X-energy Xe-100 reactors require \gls{HALEU}, but the NuScale VOYGR requires
a similar enrichment level to current \gls{LWR} fuel (Table \ref{tab:reactor_summary}). 
The NuScale VOGYR was included in this work, despite not requiring \gls{HALEU} fuel, 
because it is close to obtaining regulatory approval and is very likely to be 
deployed along-side reactors that require \gls{HALEU}. 

\begin{table}[ht]
    \centering
    \caption{Advanced reactor design specifications. }
    \label{tab:reactor_summary}
    \begin{tabular}{p{5cm}p{3cm}p{3cm}p{3cm}}
        \hline
        Design Criteria & \gls{USNC} \gls{MMR} & 
            X-energy Xe-100 & NuScale VOYGR \\\hline
        Reactor type & Modular HTGR & Modular HTGR & SMR\\
        Power Output (MWth) & 15 & 200 & 160\\
        Enrichment (\% $^{235}U$) & 13 & 15.5 & 4.55\\
        Cycle Length (years) & 20 & online refuel & 2\\
        Fuel form & \gls{TRISO} compacts & \gls{TRISO} pebbles & UO$_2$ pellets\\
        (Cycle or discharge?) Fuel burnup (GWd/MTU) & 42.7& 163& 12 (cycle) \\
        Reactor Lifetime & 20 years & 60 years & 60 years\\
        \hline
    \end{tabular}
\end{table}

Fresh fuel recipes for these reactors are based on the required fuel form for 
each reactor using the appropriate uranium isotope ratio for enrichment. Spent fuel 
recipes were obtained by performing depletion 
calculations in serpent \cite{leppanen_serpent_2014}.\hl{add more when available}

\subsection{Once-through fuel cycle}
The flow of material through the modeled once-through fuel cycles is shown 
in Figure \ref{fig:once-through_fuel_cycle}. The ``advanced reactor'' node in Figure 
\ref{fig:once-through_fuel_cycle} represents any subset of the advanced reactors included 
in the scenario. The \glspl{LWR} are deployed at their specific start and end dates 
using the ``cycamore::DeployInst'', the advanced reactors are deployed as needed by 
the ``cycamore::ManagerInst'', and the energy demand of the scenario is defined using 
the ``cycamore::GrowthRegion'' \cite{scopatz_cyclus_2015}.

\begin{figure}
    \centering
    \begin{tikzpicture}[node distance=1.5cm]
        \node (mine) [facility] {Uranium Mine};
        \node (mill) [facility, below of=mine] {Uranium Mill};
        \node (conversion) [facility, below of=mill] {Conversion};
        \node (enrichment) [facility, below of=conversion]{Enrichment};
        \node (fabrication) [facility, below of=enrichment]{Fuel Fabrication};
        \node (reactor) [facility, below of=fabrication]{Reactor};
        \node (adv_reactor) [transition, right of=reactor, xshift=3cm]{Advanced Reactor};
        \node (wetstorage) [facility, below of=reactor]{Wet Storage};
        \node (drystorage) [facility, below of=wetstorage]{Dry Storage};
        \node (sinkhlw) [facility, below of=drystorage, xshift=5cm]{HLW Sink};
        \node (sinkllw) [facility, right of=enrichment, xshift=3cm]{LLW Sink};

        \draw [arrow] (mine) -- node[anchor=east]{Natural U} (mill); 
        \draw [arrow] (mill) -- node[anchor=east]{U$_3$O$_8$}(conversion); 
        \draw [arrow] (conversion) -- node[anchor=east]{UF$_6$}(enrichment);
        \draw [arrow] (enrichment) -- node[anchor=east]{Enriched U}(fabrication);
        \draw [arrow] (enrichment) -- node[anchor=south]{Tails}(sinkllw);
        \draw [arrow] (fabrication) -- node[anchor=east]{Fresh UOX}(reactor);
        \draw [arrow] (fabrication) -- node[anchor=west]{TRSIO fuel}(adv_reactor);
        \draw [arrow] (reactor) -- node[anchor=east]{Spent UOX}(wetstorage);
        \draw [arrow] (wetstorage) -- node[anchor=east]{Cool Spent UOX}(drystorage);
        \draw [arrow] (drystorage) -- node[anchor=east]{Casked Spent UOX}(sinkhlw);
        \draw [arrow] (adv_reactor) -- node[anchor=west]{Spent TRISO Fuel}(sinkhlw);

        \end{tikzpicture}
    \caption{Fuel cycle facilities and material flow between facilities. Facilities in 
    red are added in for the transition scenarios.}
    \label{fig:fuel_cycle}
\end{figure}


The transition scenarios investigated model the transition to multiple 
combinations of these reactors and different energy demand scenarios, 
summarized in Table \ref{tab:scenarios_once-through}

\begin{table}[ht]
    \centering
    \caption{Summary of the once-through fuel cycle transition scenarios.}
    \label{tab:scenarios_once-through}
    \begin{tabular}{l l l}
            \hline
            Scenario Number & Reactors Present & Energy growth model\\\hline
            1 & \glspl{LWR} & N/A \\
            2 & \glspl{LWR} and \gls{USNC} \gls{MMR} & No growth \\
            3 & \glspl{LWR} and X-energy Xe-100& No growth \\
            4 & \glspl{LWR}, X-energy Xe-100, \gls{USNC} \gls{MMR}& No growth\\
            5 & \glspl{LWR}, \gls{USNC} \gls{MMR}, NuScale VOYGR & No growth\\
            6 & \glspl{LWR}, X-energy Xe-100, NuScale VOYGR & No growth\\
            7 & \glspl{LWR}, X-energy Xe-100, \gls{USNC} \gls{MMR}, NuScale VOYGR & No growth\\
            8 & \glspl{LWR} and \gls{USNC} \gls{MMR}& 1\% growth \\
            9 & \glspl{LWR} and X-energy Xe-100& 1\% growth\\
            10 & \glspl{LWR}, X-energy Xe-100, \gls{USNC} \gls{MMR}& 1\% growth\\
            11 & \glspl{LWR}, \gls{USNC} \gls{MMR}, NuScale VOYGR & 1\% growth\\
            12 & \glspl{LWR}, X-energy Xe-100, NuScale VOYGR & 1\% growth\\
            13 & \glspl{LWR}, X-energy Xe-100, \gls{USNC} \gls{MMR}, NuScale VOYGR & 1\% growth\\
            \hline
    \end{tabular}
\end{table}

\subsection{Recycle fuel cycle}
The flow of material through the fuel cycle scenarios with recycling is shown in 
Figure \ref{fig:recycle_fuel_cycle}. Only spent fuel from the advanced reactors 
is reprocessed and reused in a reactor. 
\begin{figure}
    \centering
    \begin{tikzpicture}[node distance=1.5cm]
        \node (mine) [facility] {Uranium Mine};
        \node (mill) [facility, below of=mine] {Uranium Mill};
        \node (conversion) [facility, below of=mill] {Conversion};
        \node (enrichment) [facility, below of=conversion]{Enrichment};
        \node (fabrication) [facility, below of=enrichment]{Fuel Fabrication};
        \node (reactor) [facility, below of=fabrication]{Reactor};
        \node (adv_reactor) [transition, right of=reactor, xshift=3cm]{Advanced Reactor};
        \node (wetstorage) [facility, below of=reactor]{Wet Storage};
        \node (drystorage) [facility, below of=wetstorage]{Dry Storage};
        \node (sinkhlw) [facility, below of=drystorage, xshift=2.5cm]{HLW Sink};
        \node (sinkllw) [facility, right of=enrichment, xshift=3cm]{LLW Sink};
        \node (separation) [transition, below of=adv_reactor, yshift=-1.5cm]{Separations};
        \node (fuelfab) [transition, below of=adv_reactor,xshift=3cm]{Fuel Fab};

        \draw [arrow] (mine) -- node[anchor=east]{Natural U} (mill); 
        \draw [arrow] (mill) -- node[anchor=east]{U$_3$O$_8$}(conversion); 
        \draw [arrow] (conversion) -- node[anchor=east]{UF$_6$}(enrichment);
        \draw [arrow] (enrichment) -- node[anchor=east]{Enriched U}(fabrication);
        \draw [arrow] (enrichment) -- node[anchor=south]{Tails}(sinkllw);
        \draw [arrow] (fabrication) -- node[anchor=east]{Fresh UOX}(reactor);
        \draw [arrow] (fabrication) -| node[anchor=west]{Fresh Fuel}(adv_reactor);
        \draw [arrow] (reactor) -- node[anchor=east]{Spent UOX}(wetstorage);
        \draw [arrow] (wetstorage) -- node[anchor=east]{Cool Spent UOX}(drystorage);
        \draw [arrow] (drystorage) |- node[anchor=east]{Casked Spent UOX}(sinkhlw);
        \draw [arrow] (adv_reactor) -- node[anchor=east]{Spent Fuel}(separation);
        \draw [arrow] (separation) -| node[anchor=west]{Separated fissile material}(fuelfab);
        \draw [arrow] (fuelfab) |- node[anchor=south]{Reprocessed fuel}(adv_reactor);
        \draw [arrow] (separation) |- node[anchor=west]{Fission Products}(sinkhlw);

        \end{tikzpicture}
    \caption{Fuel cycle facilities and material flow between facilities. Facilities in 
    red are added in for the transition scenarios.}
    \label{fig:recycle_fuel_cycle}
\end{figure}


