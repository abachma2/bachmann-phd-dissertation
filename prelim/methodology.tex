\section{Data}
Data for LWR assembly masses was pulled from 

[1] N. Todreas and M. Kazimi, 
Nuclear systems. Boca Raton, FL: CRC Press, 2012.

[2] D. Cacuci, Handbook of nuclear engineering. New York: Springer, 2010.

\section{Simulations}
This work employ \Cyclus \cite{huff_fundamental_2016} to simulate each fuel 
cycle. This allows for deployment of reactors to meet any required energy 
demand, as well as material transactions between facilities. Each scenario
models a once-through fuel cycle, with the primary difference being the 
reactor types deployed. Figure \ref{fig:fuel_cycle} shows the facilities 
and materials in the scenarios. Facilities in red are used in scenarios 
with \gls{HALEU} fueled reactors, and provide a separation between the 
\gls{LEU} and \gls{HALEU} material streams. 

\begin{figure}
    \centering
    \begin{tikzpicture}[node distance=1.5cm]
        \node (mine) [facility] {Uranium Mine};
        \node (mill) [facility, below of=mine] {Uranium Mill};
        \node (conversion) [facility, below of=mill] {Conversion};
        \node (enrichment) [facility, below of=conversion, xshift=-5cm]{Enrichment};
        \node (enrichment2) [transition, below of=conversion, xshift=5cm]{Enrichment};
        \node (fabrication2) [transition, below of=enrichment2,yshift=-1cm]{Fuel Fabrication};
        \node (fabrication) [facility, below of=enrichment, yshift=-1cm]{Fuel Fabrication};
        \node (reactor) [facility, below of=fabrication]{Reactor};
        \node (adv_reactor) [transition, below of=fabrication2]{Advanced Reactor};
        \node (wetstorage) [facility, below of=reactor]{Wet Storage};
        \node (drystorage) [facility, below of=wetstorage]{Dry Storage};
        \node (sinkhlw) [facility, below of=drystorage, xshift=5cm]{HLW Sink};
        \node (sinkllw) [facility, below of=enrichment, xshift=5cm]{LLW Sink};

        \draw [arrow] (mine) -- node[anchor=east]{Natural U} (mill); 
        \draw [arrow] (mill) -- node[anchor=east]{U$_3$O$_8$}(conversion); 
        \draw [arrow] (conversion) -- node[anchor=east]{UF$_6$}(enrichment);
        \draw [arrow] (enrichment) -- node[anchor=east]{Enriched U}(fabrication);
        \draw [arrow] (conversion) -- node[anchor=west]{UF$_6$}(enrichment2);
        \draw [arrow] (enrichment2) -- node[anchor=west]{HALEU}(fabrication2);
        \draw [arrow] (enrichment) -- node[anchor=east]{Tails}(sinkllw);
        \draw [arrow] (enrichment2) -- node[anchor=west]{Tails}(sinkllw);
        \draw [arrow] (fabrication) -- node[anchor=east]{Fresh UOX}(reactor);
        \draw [arrow] (fabrication2) -- node[anchor=west]{TRSIO fuel}(adv_reactor);
        \draw [arrow] (reactor) -- node[anchor=east]{Spent UOX}(wetstorage);
        \draw [arrow] (wetstorage) -- node[anchor=east]{Cool Spent UOX}(drystorage);
        \draw [arrow] (drystorage) -- node[anchor=east]{Casked Spent UOX}(sinkhlw);
        \draw [arrow] (adv_reactor) -- node[anchor=west]{Spent TRISO Fuel}(sinkhlw);

        \end{tikzpicture}
    \caption{Fuel cycle facilities and material flow between facilities. Facilities in 
    red are added in for the transition scenarios.}
    \label{fig:fuel_cycle}
\end{figure}

Multiple scenarios are considered in this work, with each scenario differing
by the type(s) of reactors and the power demand. The first scenario considered 
is the current fuel cycle in the U.S. starting in 1965. This fuel cycle 
consists of only \gls{LWR}s
and has no specified power demand. This scenario provides a baseline for 
the resource requirements that are currently used. Reactors deployed in this 
scenario are the reactors that have been deployed in real life, with start and 
end dates for each reactor obtained from the \gls{IAEA} \gls{PRIS} database 
\cite{noauthor_power_1989}. Reactors still in operation in December 2020, and thus 
lacking an end date in the \gls{PRIS} database are assumed to operate for 60 
years. Only reactors with a power level above 400 MWe were used in the simulation 
to avoid including prototype and research reactors present in the database. 
Approximate masses for fuel used in the core the \gls{LWR}s was obtained 
from \cite{todreas_nuclear_2012} and \cite{cacuci_handbook_2010}. 

The other scenarios model a transition scenario that includes an advanced 
reactor requiring \gls{HALEU} fuel. Two assumptions are used about the growth 
in nuclear energy demand: one assuming no growth and the other assuming 
1\% annual growth. The transition to the new reactor type starts in 2025. 
These scenarios use the same  
non-reactor facilities as the simulation of the current 
U.S. fuel cycle. \gls{HALEU} fuel reactors 
considered include the \gls{USNC} \gls{MMR} \textsuperscript{TM}
\cite{mitchell_usnc_2020} and the X-Energy Xe-100 \textsuperscript{TM} 
Reactor \cite{harlan_x-energy_2018}\cite{hussain_advances_2018}. Both of 
these reactors are designed 
to use \gls{HALEU} fuel in the form of \gls{TRISO} fuel particles. Table 
\ref{tab:reactor_summary} summarizes important design aspects of these two reactors.

\begin{table}[ht]
    \caption{Mico-reactor design specifications}
    \label{tab:reactor_summary}
    \begin{tabular}{p{2.5cm}p{2.25cm}p{2.5cm}}
        \hline
        Design Criteria & \gls{USNC}\gls{MMR}\textsuperscript{TM} & 
            X-Energy Xe-100\textsuperscript{TM} \\\hline
        Reactor type & Modular HTGR & Modular HTGR \\
        Power Output (MWth) & 15 & 200 \\
        Enrichment (\% $^{235}U$) & 13 & 15.5 \\
        Cycle Length (years) & 20 & online refuel\\
        Fuel form & \gls{TRISO} compacts & \gls{TRISO} pebbles\\
        Reactor Lifetime & 20 years & 60 years \\
        Coolant & He & He \\
        \hline
    \end{tabular}
\end{table}
    
We selected these two reactors because they have a high 
likelihood of being deployed and because published information is 
available about their designs. Comparing the transitional \gls{HALEU} 
demands of these reactors will also inform the fuel cycle support 
implications of deploying reactors with long cycle 
times versus those utilizing online refueling. 

Each of the transition scenarios will be simulated using a single type of 
advanced reactor, resulting in four additional fuel cycle scenarios and five 
scenarios in total. The \gls{LWR}s in all of the simulations are deployed 
at their respective time steps using a ``cycamore:DeployInst'' institution, 
and the advanced recators are deployed as needed by a ``cycamore:ManagerInst''
institution. Both of these institutions are within the same 
``cycamore:GrowthRegion'' region, with a specified power demand starting at 
time step 709 (the start of year 2025) baed on the transition scenario. 