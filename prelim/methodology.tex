The fuel cycle scenarios created show the resources required for the transition from the 
current fleet of \glspl{LWR} in the US to advanced reactors using \gls{HALEU}.
All fuel cycle models were run in \Cyclus 
\cite{huff_fundamental_2016}. Each simulation models current 
\glspl{LWR} starting in 1965 and models all reactors out to 2090 with a timestep 
of one month. Start 
and select end dates for \glspl{LWR} were obtained from the \gls{IAEA} \gls{PRIS} database 
\cite{noauthor_power_1989}. Reactors still in operation in December 2020, and thus 
lacking an end date in the \gls{PRIS} database are assumed to operate through their 
current operating license 
\cite{noauthor_us_nodate}.
Only reactors with a power level above 400 MWe were used in the simulation 
to avoid including prototype and research reactors present in the database. 
Approximate masses for fuel used in the core the \glspl{LWR} was obtained 
from \cite{todreas_nuclear_2012} and \cite{cacuci_handbook_2010}. 

Material compositions for each commodity in the models are defined using recipes.
Recipes for \gls{LWR} fresh and spent fuel were found in \cite{jacobson_verifiable_2010}.
Obtaining recipes for fresh and spent fuel for the advanced reactors is described in 
Section \ref{sec:reactor_methods}. The tails assay from enrichment is defined as 
0.2\% assay. 

Fuel cycle scenarios are grouped into once-through fuel cycles and recycle fuel 
cycles. Within each group, fuel cycle models are defined based on the energy demand 
of the scenario (either no growth or 1\% annual growth in demand) and by the 
advanced reactors deployed in the scenario. The transition from \glspl{LWR} to 
advanced reactors is modeled as beginning in 2025, so the energy demand is based on the 
energy supplied by \glspl{LWR} in 2025. Many of the current planned \gls{HALEU}-fueled 
reactors are not scheduled to be deployed until \hl{???}, but 2025 was selected as 
the transition start time for this work because it will provide an upper 
bounding case for if these reactors were to be deployed in mass on 
an aggressive timeline. 

Each of the fuel cycle scenarios 
defined in this work are compared on the number of reactors deployed, the 
uranium usage (both uranium ore and what is sent to the reactors), the enrichment 
\gls{SWU} capacity required, and the amount of waste produced. The natural 
uranium usage and waste production are two of the metrics used in the \gls{ES} 
\cite{wigeland_nuclear_2014} and provide a way to compare this work to that study.

\section{Reactors} \label{sec:reactor_methods}
This work considers three advanced reactors: the \gls{USNC} \gls{MMR}, 
\cite{mitchell_usnc_2020}, the X-energy Xe-100 
\cite{harlan_x-energy_2018,hussain_advances_2018}, and the NuScale VOYGR reactor
\cite{nuscale_chapter_2020,nuscale_chapter_2020-1}. The \gls{USNC} \gls{MMR}
and X-energy Xe-100 reactors require \gls{HALEU}, but the NuScale VOYGR requires
a similar enrichment level to current \gls{LWR} fuel (Table \ref{tab:reactor_summary}). 
The NuScale VOGYR was included in this work, despite not requiring \gls{HALEU} fuel, 
because it is close to obtaining regulatory approval and is very likely to be 
deployed along-side reactors that require \gls{HALEU}. Including the NuScale 
VOYGR reactor in the transition scenarios provides insight into the material 
requirements of deploying \gls{HALEU} and non-\gls{HALEU}-fueled reactors 
in tandum. However, because the NuScale VOYGR reactor does not require 
\gls{HALEU}, the transition from \glspl{LWR} to only the VOYGR reactor is 
not considered in this work.

\begin{table}[ht]
    \centering
    \caption{Advanced reactor design specifications.}
    \label{tab:reactor_summary}
    \begin{tabular}{l p{3.5cm}p{3cm}p{3cm}p{3cm}}
        \hline
        Design Criteria & \gls{USNC} \gls{MMR} & 
            X-energy Xe-100 & NuScale VOYGR \\\hline
        Reactor type & Modular HTGR & Modular HTGR & SMR\\
        Power Output (MWth) & 15 & 200 & 160\\
        Enrichment (\% $^{235}U$) & 13 & 15.5 & 4.09\\
        Cycle Length (years) & 20 & online refuel & 2\\
        Fuel form & UO$_2$ + UCO pebbles & UCO pebbles & UO$_2$ pellets\\
        Discharge fuel burnup (GWd/MTU) & 42.7& 160 & 12 (cycle) \\
        Reactor Lifetime & 20 years & 60 years & 60 years\\
        \hline
    \end{tabular}
\end{table}

Defining a reactor with the ``cycamore:Reactor'' archetype \cite{scopatz_cyclus_2015}, 
requires the refueling scheme and the fuel recipes. The refueling scheme includes 
the cycle length, the refuling length, and the amount of fuel put into the reactor 
at each refueling. 
The \gls{MMR} is not meant to undergo refueling, the initial core is designed 
to operate for the entire lifetime of the reactor \cite{mitchell_usnc_2020}. 
Therefore, this reactor 
is modeled to not include any refueling. The mass of urnaium in the core was
calculate based on the power level, fuel burnup, and lifetime of the 
fuel in the core using values reported in \cite{hawari_development_2018}. 
The total fuel mass was then calculated based on the composition of the fuel. 
Only uranium-containing fuel components were considered in this calculation, 
Any silicon-carbide in the \gls{TRISO} particles was not 
considered, only the uranium dioxide and uranium carbide in the \gls{TRISO}
particle. Doing this calculation assumes that uranium-contaning materials would 
be the limiting factor in the fuel fabrication process and that other materials 
would be available as needed. 

The Xe-100 reactor is designed to 
undergo online refueling operations, with each \gls{TRISO} pebble passing 
through the reactor six times before discharge \cite{hussain_advances_2018}. 
Every six months about 1/7th 
of the pebbles in the core are expected to be discharged. Online refueling 
cannot be explicitly modeled in \Cyclus, so the Xe-100 refueling is modeled 
as a replacement of 1/7th of the enrtire core every 6 months. The mass of 
each \gls{TRISO} was calculated by calculating the volume of a single 
uranium carbide \cite{harlan_x-energy_2018} and multiplying by the density of the 
particle. This mass was then multiplied by the number of particles in each 
pebble, and by the number of pebbles in the core \cite{harlan_x-energy_2018}.
This calculation provided the total mass of uranium carbide in the core. The 
total mass was then divided by 7 to get an average for the fuel mass replaced 
each modeled refueling.  

The VOYGR reactor contains 37 fuel assemblies, with three different enrichment 
levels \cite{nuscale_chapter_2020}. Each refueling replaces 13 fuel assemblies, 
because the middle assembly is replaced at every outage. The mass of fuel in 
each assembly is reported in \cite{nuscale_chapter_2020}. The reported enrichment 
for the VOYGR reactor in Table \ref{tab:reactor_summary}is the average for 
each refueling. The VOYGR reactor is 
modeled as a replacement of 13 fuel assemblies every refueling outage. Each refueling 
outage is modeled as taking one month, because that is the refueling outage time 
modeled for the \glspl{LWR}, and one month is the minimum time step in the 
simulations. 

Fresh fuel recipes for these reactors are based on the required fuel form for 
each reactor using the appropriate uranium isotope ratio for enrichment. Spent fuel 
recipes were obtained from depletion 
calculations in serpent \cite{leppanen_serpent_2014}. All serpent models used 
the JEFF 3.1.2 cross section data library \cite{koning_status_2011}.
The model created for the \gls{MMR} is based on the work reported in 
\cite{hawari_development_2018}, and a cross section of the core is shown in 
\hl{FIGURE}.  \hl{add more when available}

\section{Once-through fuel cycle}
The flow of material through the modeled once-through fuel cycles is shown 
in Figure \ref{fig:once-through_fuel_cycle}. The once-through fuel cycles model 
material from the mine through final disposal in a final respository (the 
``HLW Sink'' in FIgure \ref{fig:once-through_fuel_cycle}. The ``advanced reactor'' node in Figure 
\ref{fig:once-through_fuel_cycle} represents any subset of the advanced reactors included 
in the scenario. The \glspl{LWR} are deployed at their specific start and end dates 
using the ``cycamore::DeployInst'', the advanced reactors are deployed as needed by 
the ``cycamore::ManagerInst'', and the energy demand of the scenario is defined using 
the ``cycamore::GrowthRegion'' \cite{scopatz_cyclus_2015}.

\begin{figure}
    \centering
    \begin{tikzpicture}[node distance=1.5cm]
        \node (mine) [facility] {Uranium Mine};
        \node (mill) [facility, below of=mine] {Uranium Mill};
        \node (conversion) [facility, below of=mill] {Conversion};
        \node (enrichment) [facility, below of=conversion]{Enrichment};
        \node (fabrication) [facility, below of=enrichment]{Fuel Fabrication};
        \node (reactor) [facility, below of=fabrication]{Reactor};
        \node (adv_reactor) [transition, right of=reactor, xshift=3cm]{Advanced Reactor};
        \node (wetstorage) [facility, below of=reactor]{Wet Storage};
        \node (drystorage) [facility, below of=wetstorage]{Dry Storage};
        \node (sinkhlw) [facility, below of=drystorage, xshift=5cm]{HLW Sink};
        \node (sinkllw) [facility, right of=enrichment, xshift=3cm]{LLW Sink};

        \draw [arrow] (mine) -- node[anchor=east]{Natural U} (mill); 
        \draw [arrow] (mill) -- node[anchor=east]{U$_3$O$_8$}(conversion); 
        \draw [arrow] (conversion) -- node[anchor=east]{UF$_6$}(enrichment);
        \draw [arrow] (enrichment) -- node[anchor=east]{Enriched U}(fabrication);
        \draw [arrow] (enrichment) -- node[anchor=south]{Tails}(sinkllw);
        \draw [arrow] (fabrication) -- node[anchor=east]{Fresh UOX}(reactor);
        \draw [arrow] (fabrication) -- node[anchor=west]{TRSIO fuel}(adv_reactor);
        \draw [arrow] (reactor) -- node[anchor=east]{Spent UOX}(wetstorage);
        \draw [arrow] (wetstorage) -- node[anchor=east]{Cool Spent UOX}(drystorage);
        \draw [arrow] (drystorage) -- node[anchor=east]{Casked Spent UOX}(sinkhlw);
        \draw [arrow] (adv_reactor) -- node[anchor=west]{Spent TRISO Fuel}(sinkhlw);

        \end{tikzpicture}
    \caption{Fuel cycle facilities and material flow between facilities. Facilities in 
    red are added in for the transition scenarios.}
    \label{fig:fuel_cycle}
\end{figure}

The once-through scenarios model the current fleet of \glspl{LWR} in the 
US and the transition to multiple 
combinations of these reactors and different energy demand scenarios, 
summarized in Table \ref{tab:scenarios_once-through}. Scenario 1 models
the \gls{LWR} fleet without the transition to any advanced reactor to provide 
a comparison to if no new reactors are built.  

\begin{table}[ht]
    \centering
    \caption{Summary of the once-through fuel cycle transition scenarios.}
    \label{tab:scenarios_once-through}
    \begin{tabular}{l l l}
            \hline
            Scenario Number & Reactors Present & Energy growth model\\\hline
            1 & \glspl{LWR} & N/A \\
            2 & \glspl{LWR} and \gls{USNC} \gls{MMR} & No growth \\
            3 & \glspl{LWR} and X-energy Xe-100& No growth \\
            4 & \glspl{LWR}, X-energy Xe-100, \gls{USNC} \gls{MMR}& No growth\\
            5 & \glspl{LWR}, \gls{USNC} \gls{MMR}, NuScale VOYGR & No growth\\
            6 & \glspl{LWR}, X-energy Xe-100, NuScale VOYGR & No growth\\
            7 & \glspl{LWR}, X-energy Xe-100, \gls{USNC} \gls{MMR}, NuScale VOYGR & No growth\\
            8 & \glspl{LWR} and \gls{USNC} \gls{MMR}& 1\% growth \\
            9 & \glspl{LWR} and X-energy Xe-100& 1\% growth\\
            10 & \glspl{LWR}, X-energy Xe-100, \gls{USNC} \gls{MMR}& 1\% growth\\
            11 & \glspl{LWR}, \gls{USNC} \gls{MMR}, NuScale VOYGR & 1\% growth\\
            12 & \glspl{LWR}, X-energy Xe-100, NuScale VOYGR & 1\% growth\\
            13 & \glspl{LWR}, X-energy Xe-100, \gls{USNC} \gls{MMR}, NuScale VOYGR & 1\% growth\\
            \hline
    \end{tabular}
\end{table}

\section{Recycle fuel cycle}
The flow of material through the fuel cycle scenarios with recycling is shown in 
Figure \ref{fig:recycle_fuel_cycle}. Only spent fuel from the advanced reactors 
is reprocessed and reused in a reactor. \hl{Do I want fuel to be disposed of 
after one additional pass? Probably need to read literature}

\begin{figure}
    \centering
    \begin{tikzpicture}[node distance=1.5cm]
        \node (mine) [facility] {Uranium Mine};
        \node (mill) [facility, below of=mine] {Uranium Mill};
        \node (conversion) [facility, below of=mill] {Conversion};
        \node (enrichment) [facility, below of=conversion]{Enrichment};
        \node (fabrication) [facility, below of=enrichment]{Fuel Fabrication};
        \node (reactor) [facility, below of=fabrication]{Reactor};
        \node (adv_reactor) [transition, right of=reactor, xshift=3cm]{Advanced Reactor};
        \node (wetstorage) [facility, below of=reactor]{Wet Storage};
        \node (drystorage) [facility, below of=wetstorage]{Dry Storage};
        \node (sinkhlw) [facility, below of=drystorage, xshift=2.5cm]{HLW Sink};
        \node (sinkllw) [facility, right of=enrichment, xshift=3cm]{LLW Sink};
        \node (separation) [transition, below of=adv_reactor, yshift=-1.5cm]{Separations};
        \node (fuelfab) [transition, below of=adv_reactor,xshift=3cm]{Fuel Fab};

        \draw [arrow] (mine) -- node[anchor=east]{Natural U} (mill); 
        \draw [arrow] (mill) -- node[anchor=east]{U$_3$O$_8$}(conversion); 
        \draw [arrow] (conversion) -- node[anchor=east]{UF$_6$}(enrichment);
        \draw [arrow] (enrichment) -- node[anchor=east]{Enriched U}(fabrication);
        \draw [arrow] (enrichment) -- node[anchor=south]{Tails}(sinkllw);
        \draw [arrow] (fabrication) -- node[anchor=east]{Fresh UOX}(reactor);
        \draw [arrow] (fabrication) -| node[anchor=west]{Fresh Fuel}(adv_reactor);
        \draw [arrow] (reactor) -- node[anchor=east]{Spent UOX}(wetstorage);
        \draw [arrow] (wetstorage) -- node[anchor=east]{Cool Spent UOX}(drystorage);
        \draw [arrow] (drystorage) |- node[anchor=east]{Casked Spent UOX}(sinkhlw);
        \draw [arrow] (adv_reactor) -- node[anchor=east]{Spent Fuel}(separation);
        \draw [arrow] (separation) -| node[anchor=west]{Separated fissile material}(fuelfab);
        \draw [arrow] (fuelfab) |- node[anchor=south]{Reprocessed fuel}(adv_reactor);
        \draw [arrow] (separation) |- node[anchor=west]{Fission Products}(sinkhlw);

        \end{tikzpicture}
    \caption{Fuel cycle facilities and material flow between facilities. Facilities in 
    red are added in for the transition scenarios.}
    \label{fig:recycle_fuel_cycle}
\end{figure}

\begin{table}[ht]
    \centering
    \caption{Summary of the recycle fuel cycle transition scenarios.}
    \label{tab:scenarios_recycle}
    \begin{tabular}{l l l}
            \hline
            Scenario Number & Reactors Present & Energy growth model\\\hline

            \hline
    \end{tabular}
\end{table}

\section{Calculation of results}
The results presented for the transition scenarios are from the SQLite database 
that is output by each \Cyclus simulation. Because the non-reactor facilities 
in the simulations (e.g. enrichment facility) do not have a specific cap on the 
amount of material they can hold, the material being traded by these facilities 
does not accurately capture the material requirements of the reactors. Therefore, 
the material requirements and waste of each scenario are calculated based on 
the material sent to and discharged from the reactors in each scenario. 

The mass of enriched uranium reported is the mass that is sent to the reactors 
at each time step. The \gls{SWU} required is also calculated based on the 
mass sent to the reactors, and does not reflect the capacity of an actual 
facility. The mass of natural uranium required to produce the enriched uranium 
is calculated based on the mass of product sent to the reactors, Eq. \ref{eq:enrichment},
and 
